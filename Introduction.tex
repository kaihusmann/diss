\setchapterpreamble[uc][0.8\textwidth]{%
	\dictum[Davis, Johnson, Bettinger, Howard, \textit{Forest Management}]{%
		``Forest management, whether for timber production, biodiversity, or any other goals, requires decisions that are based on both our knowledge of the world and human values.''}\vskip1em}

\chapter{Introduction}
\label{chap:intro}
\newpage
\section{Decision support systems in forest planning}
\label{sec:intro:dss}
Decision making is the last step in the process of planning, starting with actually discovering the existence of a decision problem. Complexity of the planning process, thereby determined through the type of decision, may vary from very simple daily decisions to extensive and long-lasting decision processes \citep[p. 3-4]{kangas_2015}. Relatively easy and quick decisions differ fundamentally from complexer decisions in terms of their problem structures, consequences, preferences of the decision-maker and solution evaluation \citep[p. 807-808]{keeney_1982}. While everyday choices in professional framework usually base on associative selections and personal preferences, crucial decisions or decisions with long-lasting consequences are often taken analytically with explicit inference (\citealp[p. 659, 672]{stanovich_2000}; \citealp[p. 3]{kangas_2015}). Those two different decision typos, often called \textit{systems} in behavioral sciences \citep[e.g.][p. 658]{stanovich_2000}, underly significantly different theorems. Following classical theory of decision examination, decisions can be described either \textit{descriptively} or \textit{normatively} \citep[p. 6]{bitz_2005}. Examining decisions from descriptive perspective means evaluating individual and social actions. The descriptive decision theory analyses actual decisions with aim of examining how decision-makers act in reality and how decision making actually works. It hence analyses the principles descriptively without further investigation of purposes. Descriptive decision studies will answer the question \textit{how} but not \textit{why} decision-makers decide \citep[p. 499-501]{simon_1979}. Findings from descriptive studies hence do not allow drawing any conclusions about plausibility or reasonability. Lessons from empiric-descriptive analysis, however, do not necessarily lack reasonability. Descriptive decision theory just not inquiries rationals behind decisions \citep[p. 500]{simon_1979}. Target of the normative decision theory is examination of particular reasons behind decisions. Hence, in contrast to the descriptive theory, normative studies are rather evidence-based than descriptive. They aim on theoretical explanation of the causal network which leads to decisions. Normative decision models are, usually computer-aided, mathematical, statistical or numerical computations, trying to explain decision processes accounting for their intrinsic criteria. If researchers or decision-makers are interested in the reasons behind fairly complex decision problems, normatively examination will usually be obligatory.

Normative decision examination forms the foundation of \textit{operations research} (\citealp[p. 112]{shim_2002}; \citealp[p. 498]{simon_1979}), an interdisciplinary science with elements from statistics, mathematics, economics and computer sciences which developed simultaneously with the first digital computers \citep{churchman_1957}. Operations research, also called management or decision sciences, is the science of building and using computer-aided models for decision support \citep[p. 373-374]{wacker_1998}, nowadays mandatory tools for most all crucial intermediate- and long-term decisions in professional framework. Operations research hence builds the theoretical background of all modern computer-aided programs for decision support, also called \textit{decision support systems} (DSS). DSS are software that structure the programmable part of a decision problem into solvable systems of variables and rules. Since their initial definition in \citet[p. 26]{gorry_1971}, it is stated that decision processes must be simplified and abstracted in a way that they can be programed as computer code. As simplification of decision processes into programmable elements is usually performed using statistical inference such as regression or variance analysis, applied statistical models are mandatory tools right from the beginning of operations research \citep{churchman_1957}. All further elements of decisions, in particular non-programmable processes that are unstructured or far too complex, cannot be considered in DSS. Prior implementation into computer-aided models, decision-relevant aspects must be gathered, reviewed and simplified. A complete, accurate normative decision analysis is hence a prerequisite for the development of DSS. Most all modern decision support models lastly base upon classical decision theory as they basically translate normatively inquiry into applicable models for scientists, practitioners or any person being responsible for decisions. DSS aim on objective, theory-based solutions of real decision problems. Next to the actual variables and rules, typical DSS therefore have a user front-end and a data-warehouse (\citealp[p. 2]{hansen_2012}; \citealp[p. 115]{shim_2002}). The user front-end facilitates application for the user, the data-warehouse enables storage of necessary input data and the solution.

Decisions in forest management will never affect things solely. Once made, management decisions will have impact on lots of economic, ecologic and social issues. Forests should be manged such that they produce income for the forest owner on the one, while contemporaneously follow conservation and recreational issues, on the other hand \citep[p. 11]{kangas_2015} \textcolor{red}{(CITE M�Hing. 1997 ODER Waldeigentum 923 Dep in BBF)}. Simultaneous fulfillment of all three functions on the entire forest land is the guideline principle of the multifunctional forestry in Germany. It is firmly anchored in the Federal Forest Law \citep[p. 457]{moller_2007}. The concept of \textit{sustainability} plays a central role in forestry. In one of its most recent and general definitions, sustainability states development of forests such that current and future generations can benefit from all three forest functions (\citealp[p. 14]{un_2005}; see also \citealp[p. 14]{kangas_2015}). In consequence, practice forest management decisions must exclude declining of several different aspects like \textit{biodiversity}, \textit{productivity} and \textit{regeneration capacity vitality} (Ministerial Conference on the Protection of Forests in Europe, cited from \citealp[p. 15]{kangas_2015}). Forest management thus requires careful planning considering multiple criteria at the same time in each decision. Owing to this high complexity, crucial decisions in practical forestry are rarely made by single persons. Forest operations planning e.g. usually bases on a complex collaboration of inventory and action planning being comprised of long-term strategies, intermediate-term management plans and short-term operational decisions. Intermediate-term forest planing typically relies on the forest mensuration which is a long-established, continuously improved planning procedure trying implementing strategic orientations into spatial and temporal explicit operations \citep[p. 156-158]{bockmann_2004}. In addition to the high degree of complexity, a further challenge foresters typically have to cope with, is the long-lasting consequence of their operations. In forestry definition, intermediate-term planning already comprises time periods of 5-20 years \citep[p. 12]{kangas_2015}. Daily operational decisions of foresters, such as harvesting on stand level or planting will usually have consequences that last much longer than any plan. Decision maker must hence review consequences of their decisions very  thoroughly. 

Next to an increasing demand for wood as raw materials \citet[p. 8]{mantau_2012}, requirements on conservation and recreational issues are rising as well. Germany's national strategy of biological diversity e.g. provides natural development on 10\% of public owned and 5\% of private owned forest land \citep[p. 45]{bmu_2007}. As consequence, more than 700,000 hectares of forest land are planned to be set-aside by 2020 \citep{ti_2014}. \citet[p. 3]{auer_2016} calculated particularly European beech wood potential in the center of Germany. They came to the conviction that potential was already almost completely used in the time period 2002 to 2012. Further increasing demand for wood, decreasing available productive area and contemporaneously increasing recreational issues \citep[p. 1]{hansen_2012} while still producing positive returns pose new challenges for the forestry and the entire wood sector. The degree of complexity in forest operations is thus expected to increase even further which explains the need for detailed, rational decision support \citep[p. 2]{hansen_2012}. If demand further rises, innovative computer-aided supply chains as well as processing schemes become mandatory to serve all wood consumers properly.

DSS can be used to structure the highly complex forest decision problems into smaller, programmable sub-problems and finally facilitate and solve many decision problems in forestry. The advantages of DSS for forestry purposes were discovered in the early 1980s \citep[p. 499]{reynolds_2008}. Nowadays, numerous examples of useful DSS can be found in forest practice and forest sciences covering a very broad range of purposes. Very simple DSS are e.g the generalized \textit{maximin} method or the \textit{certainty equivalent} method \citep[p. 25, 28]{kangas_2015}. Complex inference based decision support tools like the \textit{WaldPlaner} \citep{hansen_2014} are already relevant part in the practical forest mensuration e.g. in Lower Saxony \citep[p. 158]{bockmann_2004}. WaldPlaner is a DSS for practitioners and scientists with user friendly interface that combines a data-warehouse with the long-established and widely used tree growth and yield simulation software \textit{Tree Growth Open Source Software} (TreeGrOSS) (\citealp[p. 46]{hansen_2014}; \citealp{nagel_2009}). It enables i.e. growth and yield simulation of multiple forest stands. It offer foresters opportunity to review consequences of management decisions and hence supports sophisticated evaluation of forestry decisions. Computer-aided forest simulation software in general has great opportunity for improvement effectiveness and accuracy of forest planning \citep[p. 210]{davis_2001}. They are advantageous against classical tables since they can consider plenty more relevant aspects for growth and yield such as tree individual concurrence, species mixtures, complex within-stand structures as well as specific thinning and usage rules (\citealp[p. 3]{hansen_2012} \citealp[p. 93]{muys_2010}).

Right from the beginning of DSS in forestry framework, optimization techniques came to use \citep[p. 16]{kangas_2015}. As optimization procedures were initially developed for efficient allocation of finite resources \citep[p. 271]{davis_2001}, they suit to mot all decision problems in practical forestry, a field with naturally scarce resources. Recently, combinations of modern growth and yield simulation software and optimization procedures are in focus of forestry research. Co-application of growth simulators and optimization procedures offers opportunities for practical production planning as it enables foresters to respect long-term changing environmental circumstances such as climate change or nitrogen deposition in their operational short-term decisions \citep[p. 1081]{pretzsch_2008}. A recent example from forest sciences can e.g. be found in \citet{yousefpour_2009}. They developed an optimization model able to optimize thinning activity in terms of economic return also taking timber production, carbon storage and biodiversity issues into account. They combine the growth and yield simulation software TreeGrOSS with the \textit{dynamic linear programming} optimization procedure. They used an integrated simulation-optimization approach to estimate the monetary drawback of different scenarios changing in their carbon storage and biodiversity characteristics. Combined simulation-optimization methods are not novel in forestry. The \textit{United States Forest Service} e.g. applies combined methods in practical forest planning since the 1970s \citep[p. 33]{hoganson_2015}. Other countries have also experience in practical application of combined methods. Recently combined methods are practically used e.g. in USA and Finland \citep[p. 41]{hoganson_2015}. Since both growth and yield simulation models as well as optimization methods have developed considerably in the last decade, modern combined methods have opportunity to solve more sophisticated decision problems in forestry (\citealp[p. 16-17]{kangas_2015}; \citealp[p. 93]{muys_2010}).

\section{The role of bio-economy in forestry}
\label{sec:intro:biecon}
Bio-economy is defined as a combination of all economic sectors that refine biologic resources with physical, chemical and biotechnological processes \citep[p. 10462]{debesi_2015}. The bio-economy itself is hence no novel sector but an aggregation of former separately regarded sectors which all base on biomass as major resource. Main advantages of bio-economy as an aggregated sector are i.a. research cooperations of former distinct industries in order to benefit from synergy \citep[p. 1]{auer_2016} as well as commonly evaluated supply chains \citep[p. 1-2]{geldermann_2016}. The cooperation is advantageous from view of forestry as it could improve possibilities of planning. If supply chains of former distinct smaller firms are comprised to a integrated super-regional supply chain including interactions between the firms, which is one aim of bio-economy, logistical planning for forest enterprises will be simplified. Supplying to a well-prepared integrated logistic system will be advantageous in terms of planning dependability and planning costs. Furthermore, from perspective of forestry, a common balance of required resources provides additional advantages in planning. Using modern simulation techniques, forecasting of wood potential is relatively easy. It is hence already often practiced in the context of cluster studies \citep[e.g.][]{bmel_2016}, scientific studies (e.g. chapter \ref{chap:hzb}) or practical mensuration \citep[e.g.][]{bockmann_2004}. Sound knowledge of resources demands for intermediate-term periods will have severe positive effects on forest planning as it enables matching the forecasted wood potential with prospective demands.

With an overall turnover of 2 trillion {\euro} in 2014, covering resources from agriculture, forestry and fishery, the European Union's bio-economy sector is leading in worldwide comparison \citep[p. 221, 223]{elchichakli_2016} with especially Germany playing an important role \citep[p. 200]{hennig_2016}. Although the share of forestry itself as primary producer amounts only to 2 \% (35 billion {\euro}), forestry plays currently an crucial role for resource supply in the bio-economy sector. The bio-based sector is a considerable factor in Germany's national economy as well. The share of bio-based economy on Germany's gross value was approximately 8 \% in 2007 including primary production as well as manufacturing and services \citep[p. 29-30]{efken_2012}.  Innovative bio-based products have opportunity to substitute end- and semifinished products which traditionally base on fossil resources. In the national bio-economy strategy, the German Government therefore decided to generally strengthen the bio-based industries by 2030 \citep[p. 15-16]{bmel_2014b}. Novel production methods could enhance significance of forest biomass, in particular of small dimensioned wood, for use in bio-refineries. \citet[p. 49]{ekman_2013} e.g. revealed woody biomass to have great potential for chemical semifinished products. Regarding the political and economical circumstances, the importance of bio-economy sector generally and in particular for forestry is thus expected to increase. From view of forestry, the question arises, whether prospective demands for woody biomass can still be served with the available resources.

\section{Aim of the thesis}
\label{sec:intro:aim}
Biomass from forests has opportunity to provide a largely carbon-neutral supply of material to the bio-based sector. In times of rising demands on all forest function, the question is \textit{''How to match the resource demands of a rising bio-economy industry with the available wood potential without compromising the concept of sustainability?''}. The answer to this question challenging as resources are limited and wood supply, from planning to distribution, are complex causal networks. I present a few applied statistical methods, some directly practically applicable, some primary of theoretical interest, trying approximate the solution of the research question from perspectives of forestry and wood industry.

\section{Structure of the thesis}
\label{sec:intro:struct}
After the general introduction, I present distinct essays to overcome the research question and the introductory stated problems. In chapter \ref{chap:discussion}, I close with a general discussion of all four studies.

\subsection{Analyzing status and development of raw wood availability in the European beech-dominated central of Germany}
\label{subsec:intro:struct:hzb}
The first step in a planning process is, as already stated, the actual identification of a problem. If resources are not scarce, they will be no distribution problems. For this, we investigated the current and future availability of European beech (\textit{Fagus sylvatica} [L.]) raw wood, one of the most important wood resource for the increasing bio-economy sector, in the beech-dominated center of Germany \citep[p. 1]{auer_2016}. Chapter \ref{chap:hzb} shows the results of a cluster study analyzing the German federal states Lower Saxony, North Rhine-Westphalia, Hesse, Saxony-Anhalt and Thuringia in terms of their beech wood potentials and demands. The raw data of the national forest inventory \citep{schmitz_2008}, a systematic large-scale sample inventory, to estimate the availability and spatial distribution of beech raw wood. The future potential was forecasted using the forest DSS WaldPlaner \citep{hansen_2014}. The study is an example of how inventory data extrapolation can be used to perform well-founded cluster analysis on a large spatial scale.

\subsection{Biomass functions and nutrient contents of European beech, oak, sycamore maple and ash and their meaning for the biomass supply chain}
\label{subsec:intro:struct:bm}
Biomass functions and nutrient contents are helpful for several decision problems in forestry and bio-economy. Potential of forest sites can only be exploited fully when exhausting of soil is excluded. Calculation of site-specific harvesting potential thus requires reliable informations on biomass and nutrients exports. Biomass functions and nutrient contents hence help calculating the timber potential for the bio-economy. The information can e.g. be used for parameterization of DSS such that biomass and nutrient contents of single trees and harvesting actions can be calculated. Biomass functions and nutrient contents are available for European beech and oak (\textit{Quercus robur} [L.] and \textit{Quercus petraea} [Matt.]) but not for sycamore maple (\textit{Acer pseudoplatanus} [L.]) and ash (\textit{Fraxinus excelsior} [L.] which are frequently in mixture with beech \citep{ti_2014}. In our essay \textit{Biomass functions and nutrient contents of European beech, oak, sycamore maple and ash and their meaning for the biomass supply chain} (chapter \ref{chap:bm}), we present a generalized nonlinear regression analysis to calculate biomass functions and nutrient contents for beech, oak, sycamore maple and ash.

\subsection{Modelling the economically viable wood in the crown of European beech trees}
\label{subsec:intro:struct:beech_crowns}
The essay in chapter \ref{chap:beech_crowns} shows higher degree of detail. In the essay, we present a model, able to predict the economically viable wood from beech crowns. It is hence a tool, helping to fully exploit the timber potential on single tree level. It is the first model in scientific forestry literature able to predict the wood volume of beech crowns with respect to the complex sympodial crown structure of broad leaf tree species. It offers opportunity for the decision makers to calculate the complete wood potential of beech trees prior harvesting. A statistical model, able to distinguish viable from unviable crown branches, using randomly sampled morphological measurements, was programmed. The results from the program are, using generalized regression, cluster analysis and linear discriminant analysis, translated into simple equations such that they can easily be implemented in any DSS.

\subsection{Flexible Global Optimization with Simulated-Annealing}
\label{subsec:intro:struct:opt}
Although optimization procedures are, in international framework, already important planning tools \citep[p. 1]{hoganson_2015} and have opportunity to support operational planning while parallel considering long-term issues \citep[p. 1081]{pretzsch_2008}, they recently play only minor role in Germany. Steadily rising demands for wood as well as for further ecosystem benefits makes forest harvesting planning in Germany increasingly more complex. For this reasons, combined simulation-optimization DSS could be promising tools for German foresters and forest scientists to support decisions of harvesting activities in terms of intensity and time. Having a closer look at optimization methods seems to be worthwhile. In chapter \ref{chap:opt}, parts of a combined simulation-optimization DSS for support of intermediate-term forest harvesting planning, specifically adopted to German characteristics, are introduced. The essay preliminary deals with the opportunities and limitations of different optimization procedures for use in forestry decision support. I state the special needs of forest growth and yield optimization and discuss suitability of different optimization approaches accordingly. I finally present an optimization procedure, able to tackle the complex output of forest simulation software and develop an explicit example.

The presented optimization procedure is part of an simulation-optimization software, which is, in its current state basically comprised of four elements (Figure \ref{fig:Introduction:flowopt}). The first element is the growth simulation which actually performs the growth and yield simulations. Tree growth and yield are simulated using TreeGrOSS, a long established single-tree based simulation software of the Northwest German Research Institute. TreeGroSS which is also the back-end the widely used forest DSS WaldPlaner (\citealp[p. 6-7]{hansen_2014}). The growth and yield simulation element is hence a stand-alone Java written software, developed by \citet{nagel_1996}. The TreeGrOSS packages, formerly known as NEWS, are advantageous for an integrated simulation-combination system since they are one of the oldest and often used growth and yield software in Germany. The data-warehouse is used to store the raw data and the results of the simulations. TreeGrOSS is compatible to a variety of data-bases \citep[p. 55]{hansen_2014}.
\textsc{\begin{figure}
		\center
		\resizebox{0.7\linewidth}{!}{% Define block styles
\tikzstyle{decision} = [diamond, draw, 
text width=4.5em, text badly centered, node distance=3cm, inner sep=0pt]
\tikzstyle{block} = [rectangle, draw, 
text width=6em, text centered, rounded corners, minimum height=4em]
\tikzstyle{line} = [draw, -latex']
\tikzstyle{cloud} = [draw, ellipse, node distance=3cm,
minimum height=2em]
\tikzstyle{db} = [cylinder, draw, shape border rotate=90, minimum height=4em, text width=6em, text centered,
aspect=0.25]


\begin{tikzpicture}[node distance = 3.5cm, auto]
    % Place nodes
\node [block] (dev) {device};
\node [block, left of=dev] (sim) {growth simulation};
\node [block, right of=dev] (opt) {optimizer};
\node [db, below of=sim, node distance=2.5cm] (db) {database};
    
% Draw edges
\path [line] (sim) -- (db);
\path [line] (db) -- (sim);
\path [line] (sim) -- (dev);
\path [line] (dev) -- (sim);
\path [line] (dev) -- (opt);
\path [line] (opt) -- (dev);

\end{tikzpicture}}
		\caption{The four basic elements of the combined simulation-optimization software.}
		\label{fig:Introduction:flowopt}
	\end{figure}
}

The interface (Figure \ref{fig:Introduction:flowopt}) links the simulation with the optimization module. It has aim to translate the TreeGrOSS in- and output into state and parameter spaces that are interpretable by optimization software. It is an R written function that internally calls the growth and yield modules of TreeGrOSS. The function can be called in any R session and hence allows TreeGrOSS based growth and yield simulations directly from R. R is a flexible statistical programing language allowing relatively easy implementation and manipulation of optimizers and additional features \citep[p. 11-12]{nash_2014}. The interface hence enables easy connection between TreeGrOSS and plenty optimization libraries.

In TreeGroSS, intensity of harvesting is determined by the difference between actual and user definable goal basal area \citep[p. 149-150]{hansen_2014}. The interface function translates all goal basal areas of all forest stands $n_{stand}$ and all simulated years $n_{years}$ of an optimization problem into a matrix with real numeric values. The resulting objective matrix, which can be passed to the interface function, is hence of dimension $n_{stands}$ x $n_{years}$, where $n_years$ can also describe discrete steps of more than one year. Within the interface function, the entires of the passed objective matrix are translated into TreeGrOSS interpretable goal basal areas. After translation, the interface function internally calls the growth and yield libraries basing on these entries. The resulting harvesting wood volumes of the TreeGrOSS simulation are stored and further processed within the interface function. The volumes are rated in terms of costs and revenues, summed and finally returned. The interface function is, in principle, a function that enables manipulation of the crucial TreeGrOSS simulation settings from R.

One of the most interesting properties of forest planning optimization is the comprehensive and straightforward definition of the sustainability principle \citep[p. 15]{kangas_2015}. Simulation-optimization DSS enable objectifying the principle of sustainability as explicitly defined computer rules. To program sustainability, I implemented two restrictions in the interface. In the last simulated year, the total standing volume, hence the sum of the standing volumes in all stands is not allowed to be lower than a predefined limit. Additionally, each distinct standing volume in the last simulated year must be above a distinct stand-specific minimum limit standing volume. The predefined limits base on growth and yield simulations under standard treatment circumstances. The interface function simply returns no valid value {\tt NA}, when a simulation result is outside the restriction limits. Amongst the maximal harvesting volumes, which are determined by the sustainability limitations, the interface also enables definition of minimum harvesting volumes in form of a further user selectable restriction. The optimization model hence enables definition of user defined annual minimum harvesting volumes for four assortments. The user can distinctively parameterize minimum annual harvesting volumes for deciduous and coniferous stem and industrial wood.

The introduced combined simulation-optimization, to conclude, bases on sequential TreeGrOSS simulations with iteratively changing thinning intensity settings. Every iteration in the optimization progress is comprised of a TreeGrOSS simulation and rating of the harvested wood volume. As the TreeGrOSS packages are a complex causal network of rules, linear and nonlinear equations, the interface returns irregular response pattern including particular undefined parts. This properties makes high demands on the optimization method. Many simpler functions are not suitable for the highly complex state space of the optimization problem. A very flexible optimization method without many assumptions on the loss function is presented and exemplary applied in chapter \ref{chap:opt}.

The combined model is another example of how applied statistical modeling can be used to strengthen the intermediate-term forest planning in times of an growing bio-economy sector. It offers opportunity to optimize intensity of harvest operations, thus enlarging harvestable wood potential for the bio-economy, without violating sustainability issues. Moreover, it can be used answering the questions whether forest enterprises with specific tree species combinations are able to comply with delivery contracts. Innovative bio-economy industries such as bio-refineries often need continuous wood supply to ensure ongoing manufacturing process \textcolor{red}{CITE}. Their success crucially depends an delivery contracts with continuous wood supply. Delivery contracts can be beneficial for both forest enterprises as well, as they facilitate intermediate-term planning. The combined simulation-optimization model enables evaluation, if annual continuous delivery contracts are actual possible without violating sustainability principles. Post-hoc sensitivity of the optimization results enables, in addition, evaluation of the benefits of delivery contracts from perspective of the forest enterprises. The difference in harvestable wood potential between the unrestricted optimization and the optimization with minimum delivery restriction can be interpreted as opportunity cost of the delivery contract. It is hence the price, a forest enterprise has to pay for the advantages in intermediate-term planning. This information can be used to find an objective trade-off between advantages and costs of annual wood delivery amount.