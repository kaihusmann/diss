\setchapterpreamble[uc][.75\textwidth]{%
	\dictum[Davis, Johnson, Howard and Bettinger, \textit{Forest Management}]{%
		``Forest management, whether for timber production, biodiversity, or any other goals, requires decisions that are based on both our knowledge of the world and human values.''}\vskip1em}

\chapter{Introduction}
\label{chap:Introduction}
Decisions in forest management never affect things solely. Once made, management decisions will have impact on lots of economic, ecologic and social issues.
...Entscheidungen (OptMeth Skript)
Nachhaltigketit
-> Es ist schwer in der FW Ents. zu treffen
Wir zeigen, wie Entscheidungen untertstützt werden könne. Insbesondere wichtig weil anford an wald steigen
Können menshcl. Entscheid. nicht ersetzen aber unterstützen
Since \citet{Carlowitz_1713} introduced the principle of sustainability to forestry, it plays a central role in there, and over the last centuries it has been further developed and extended. To achieve and maintain sustainability in its different specifications \citep{Speidel_1984, Schanz_1996} can be seen as one of the main goals or even the main goal of forestry. A prerequisite for such a sustainable forestry is information on the forest resources, their conditions and changes. This information is usually gained through forest inventories.

\section{The role of bio-economy in foresty}
Bio-economy is defined as
which is not novel in forestry
it means novel industries
meaning today
outlook
Forschungsfrage: Mittel, die gestiegenen Anforderungen an den Wald zu bewaeltigen. Wie koennen die Potenziale des Waldes ausgeschoepft werden, ohne die Nachhaltigkeit zu gefaehrden?
\textsc{\begin{figure}
	\center
	\resizebox{0.7\linewidth}{!}{% Define block styles
\tikzstyle{decision} = [diamond, draw, 
text width=5.5em, text badly centered, node distance=3cm, inner sep=0pt]
\tikzstyle{block} = [rectangle, draw, 
text width=8em, text centered, rounded corners, minimum height=6em]
\tikzstyle{line} = [draw, -latex']
\tikzstyle{cloud} = [draw, ellipse, node distance=10cm,
minimum height=2em]
\tikzstyle{db} = [cylinder, draw, shape border rotate=90, minimum height=6em, text width=8em, text centered,
aspect=0.25]


\begin{tikzpicture}[node distance = 4cm, auto]
    % Place nodes
\node [block] (dev) {interface};
\node [block, left of=dev] (sim) {growth and yield simulation};
\node [block, right of=dev] (opt) {optimizer};
\node [db, below of=sim, node distance=3.3cm] (db) {data-\\warehouse};
    
% Draw edges
\path [line] (sim) -- (db);
\path [line] (db) -- (sim);
\path [line] (sim) -- (dev);
\path [line] (dev) -- (sim);
\path [line] (dev) -- (opt);
\path [line] (opt) -- (dev);

\end{tikzpicture}}
	\caption{Example flowchart.}
	\label{fig:Introduction:flowopt}
\end{figure}
}

Reference example to chapter \ref{chap:hzb}.
