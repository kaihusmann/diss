\setchapterpreamble[uc][.75\textwidth]{%
	\dictum[Davis, Johnson, Howard and Bettinger, \textit{Forest Management}]{%
		``Forest management, whether for timber production, biodiversity, or any other goals, requires decisions that are based on both our knowledge of the world and human values.''}\vskip1em}

\chapter{Introduction}
\label{chap:intro}
Decision making is the last step in the process of planning, starting with actually discovering the existence of a decision problem. Complexity of the planning process, thereby determined through the type of decision, may vary from very simple daily decisions to extensive and long-lasting decision processes \citep[p. 3-4]{kangas_2015}. Relatively easy and quick decisions differ fundamentally from complexer decisions in terms of their problem structures, consequences, preferences of the decision-maker and solution evaluation \citep[p. 807-808]{keeney_1982}. While everyday choices in professional framework usually base on associative selections and personal preferences, crucial decisions or decisions with long-lasting impacts are often taken analytically with explicit inference (\citealp[p. 659, 672]{stanovich_2000}; \citealp[p. 3]{kangas_2015}). Those two different typos, often called \textit{systems}, in behavioral sciences \citep[e.g.][p. 658]{stanovich_2000}, underly significantly different theorems. Following classical theory of decision examination, decisions can be described either \textit{descriptively} or \textit{normatively} \citep[p. 6]{bitz_2005}. Examining decisions from descriptive perspective means evaluating individual and social actions. Descriptive theory analyses actual decisions with aim of examining how decision-makers act in reality and how decision making actually works. It analyses the principles of decision making descriptively without further investigation of purposes. Descriptive decision studies will answer the question \textit{how} but not \textit{why} decision-makers decide \citep[p. 499-501]{simon_1979}. Findings from descriptive studies hence do not allow drawing any conclusions about plausibility or reasonability. Lessons from empiric-descriptive analysis, however, do not necessarily lack reasonability. Descriptive decision theory just not inquiries rationals behind decisions \citep[p. 500]{simon_1979}. Target of the normative decision theory is examination of particular reasons behind actual decisions. Hence, in contrast to the descriptive theory, normative studies are rather evidence-based than descriptive. They aim on theoretical explanation of the causal network which leads to decisions. Normative decision models are, usually computer-aided, mathematical, statistical or numerical computations trying to explain decision processes accounting for their intrinsic criteria. Whenever decision-makers are interested in the reasons of fairly complex decision problems, normatively examination will usually be obligatory. Normative decision examination forms the foundation of \textit{operations research} (\citealp[p. 112]{shim_2002}; \citealp[p. 498]{simon_1979}), an interdisciplinary discipline with elements from statistics, mathematics, economics and computer sciences which developed simultaneously with the first digital computers \citep{churchman_1957}. Operations research, also called management or decision sciences, is the science of building and using computer-aided models for decision support \citep[p. 373-374]{wacker_1998}, nowadays mandatory tools for intermediate- and long-term decisions in professional framework. Operations research hence builds the theoretical background of all modern computer-aided programs for decision support, also called \textit{decision support systems} (DSS). DSS are computer system of crucial variables and rules that structure the programmable part of a decision problem. Since the initial definition of DSS in \citet[p. 26]{gorry_1971}, it is stated that decision processes must be simplified into programmable elements. All further processes, in particular processes that are unstructured or far too complex, cannot be considered in DSS. Prior to implementing statistical computer-aided models, decision-relevant aspects must be gathered, reviewed and simplified. Only after successful identification of programmable and meaningful decision-relevant aspects, the gathered information can be used to built a statistical computer-aided model. A complete, accurate normative decision analysis is hence a prerequisite for the development of DSS. Most all modern decision support models lastly base upon classical decision theory as they basically translate normatively inquiry into applicable models for scientists, practitioners or any person being responsible for decisions. DSS aim on objective, theory-based solutions of real decision problems. Next to the actual variables and rules which are translated into program code, typical DSS therefore have a user front-end and a data-warehouse (\citealp[p. 2]{hansen_2012}; \citealp[p. 115]{shim_2002}). The user front-end facilitates application for the user, the data-warehouse enables storage of necessary input data and the solution.

Decisions in forest management will never affect things solely. Once made, management decisions will have impact on lots of economic, ecologic and social issues. Forests should be manged such that they produce income for the forest owner (CITE M�Hing. 1997) on the one, while contemporaneously follow conservation and recreational issues, on the other hand \citep[p. 11]{kangas_2015}. Simultaneous fulfillment of all 3 functions is the guideline principle of the multifunctional forestry in Germany. It is firmly anchored in the Federal Forest Law \citep[p. 457]{moller_2007}. The concept of \textit{sustainability} plays a central role in forestry. In one of its most recent and general definitions sustainability states development of forests such that current and future generations can benefit from all 3 forest functions (\citealp[p. 14]{un_2005}; see also \citealp[p. 14]{kangas_2015}). In consequence, practice forest management decisions must exclude declining of several different aspects like \textit{biodiversity}, \textit{productivity} and \textit{regeneration capacity vitality} (Ministerial Conference on the Protection of Forests in Europe, cited from \citealp[p. 15]{kangas_2015}). Forest management thus requires careful planning considering multiple criteria. Due to this complexity, decisions in practical forestry are rarely made by single persons. Forest operations planning usually bases on a complex collaboration of inventory and action planning being comprised of long-term strategies, intermediate-term management plans and short-term operational decisions. Intermediate-term forest planing typically relies on forest mensuration which is a long-established, continuously improved planning procedure trying implementing strategic orientations into spatial and temporal explicit operations \citep[p. 156-158]{bockmann_2004}. In addition to the high degree of complexity, further challenges foresters typically have to cope with, are the long-lasting consequences of their operations. Intermediate-term planning already comprises time periods of 5-20 years \citep[p. 12]{kangas_2015}. Daily operational decisions of foresters, such as harvesting on stand level or planting will usually have consequences that last much longer than any plan. Next to an increasing demand for wood as raw materials \citet[p. 8]{mantau_2012}, requirements on conservation and recreational issues are rising as well. Germany's national strategy of biological diversity e.g. provides natural development on 10\% of public owned and 5\% of private owned forests \citep[p. 45]{bmu_2007} which means that more than 700,000 hectare of forest land is to be set-aside by 2020 \citep{ti_2014}. Increasing demand for wood, decreasing available productive area and contemporaneously increasing recreational issues \citep[p. 1]{hansen_2012} pose new challenges for the forestry and wood sector. The degree of complexity in forest operations is thus expected to increase even further. This high degree of complexity explains the need for detailed, rational decision support \citep[p. 2]{hansen_2012}. DSS can be used to structure the highly complex decision problems into smaller, programmable sub-problems and finally solve many decision problems in forest practice and sciences.

The advantages of DSS for forestry purposes were already discovered in the early 1980s \citep[p. 499]{reynolds_2008}. Nowadays, numerous examples of useful DSS can be found in forest practice and forest sciences covering a very broad range of purposes. Very simple DSS in forestry are e.g the generalized maximin method or the certainty equivalent method \citep[p. 25, 28]{kangas_2015}. Complex inference based decision support tools like the \textit{WaldPlaner} (TreeGrOSS) \citep{hansen_2014} are e.g. already integral part in the forest mensuration of Lower Saxony \citep[p. 158]{bockmann_2004}. Computer-aided forest simulation software in general has great opportunity for improvement effectiveness and accuracy of forest planning \citep[p. 210]{davis_2001}. They are advantageous against classical tables since they can consider much more relevant aspects for growth and yield such as tree individual concurrence, species mixtures, complex within-stand structures as well as specific thinning and usage rules (\citealp[p. 3]{hansen_2012} \citealp[p. 93]{muys_2010}). Right from the beginning of forest DSS, optimization techniques came to use \citep[p. 16]{kangas_2015}. As optimization procedures were initially developed for efficient allocation of finite resources \citep[p. 271]{davis_2001} 

, they suit 

. Modern Growth and yield simulation software offer new possibilites

... Behandlungsoptimierung wird m�glich Schon dieersten DSS waren Optimierungsmodelle
They are hence interesting for most every thinkable decision in forestry framework. Forest growth simulation based optimization models initially appeared in the 1980s. Early models were rather if-then calculation than 

\section{The role of bio-economy in foresty}
\label{sec:intro:biecon}
Bio-economy is defined as
which is not novel in forestry
Bio-economy is on explanation of increasing demand for wood...
Changes in government energy policy and the development of new technology led to development of new markets, in particular for small dimensioned wood \citep{geldermann_2016, mccormick_2013}.
it means novel industries
meaning today
outlook

Die Entwicklung neuer Gesch�ftsfelder ... Abw�gung. Neue M�glichkeiten/Chances ... immer auch neue Einschr�nkungen/Herausforderungen die Frage ist, ob die Vorteile �berwiegen, also braucht man DSS

\section{Aim(s?) of the thesis}
\label{sec:intro:aim}


Optimization methods in particular have great opportunity 

Despite all mentioned advantages, inference based decision support tools play only minor role in Germany. While forest growth and yield simulation softwares are  applied in forest mensuration 



, formally optimization procedures are recently not part of practical forest planning.

Seht kurzes Kapitel (5-10 Zeilen) mit der Forschungsfrage
Forschungsfrage: Mittel, die gestiegenen Anforderungen an den Wald zu bewaeltigen. Wie koennen die Potenziale des Waldes ausgeschoepft werden, ohne die Nachhaltigkeit zu gefaehrden?


\section{Structure of the thesis}
\label{sec:intro:struct}

\subsection{Analyzing status and development of raw wood availability in the European beech dominated central of Germany}
\label{subsec:intro:struct:hzb}
\textbf{Chapter \ref{chap:hzb}} is an example for a boring cluster analysis.

\subsection{Biomass functions and nutrient contents of European beech, oak, sycamore maple and ash and their meaning for the biomass supply chain}
\label{subsec:intro:struct:bm}
\textbf{Chapter \ref{chap:bm}} is boring again.

\subsection{Modelling the economically viable wood in the crown of European beech trees}
\label{subsec:intro:struct:beech_crowns}
\textbf{Chapter \ref{chap:beech_crowns}} is more interesting.

\subsection{Flexible Global Optimization with Simulated-Annealing}
\label{subsec:intro:struct:opt}
Optimization is useful to program sustainability \citep[p. 15]{kangas_2015}
OM eignen sich gut f�r die Entscheidungsunterst�tzung, weil sie mehrere Kriterien ber�cksichtigen k�nnen. ... Unsere Anwendung optimier die �kon. Funkt. und ber�cksichtig Nachhaltigkeit als ...

Although the advantages of Decision Support Systems (DSS) are widely known and often discussed in forest-science literature, optimization procedures are recently seldom used in Germany. The forest management nowadays widely bases on expert knowledge and decades of experience. The strategic forest treatment is usually ruled by very detailed and long established forest programs. As the demands for wood as well as for further ecosystem benefits is steadily rising, forest planning becomes increasingly more complex. Looking at formal optimization techniques for decision support of forest activities, as they are already performed in foreign countries, therefore seems to be worthwhile. We present an optimization software as DSS for forest thinning activities on operational level which is specifically adopted to German characteristics.

According to the typically categorization of forest optimization models, we present a \textit{type-1} optimization software. Our software basically consists of 3 elements. The forest growth simulation module, the optimization module and the data warehouse. The tree growth is simulated with the TreeGrOSS which is an long established single-tree based growth and treatment simulation software of the Northwest German Research Institute (NW-FVA). The actual optimization bases on sequential TreeGROSS simulations with iteratively changing thinning intensity settings. The response of every iteration is calculated via dimension and tree species specific rating of the simulated harvested wood volume. To ensure sustainability of the simulations, we defined 2 restrictions. The total standing volume of all stands as well as the standing volume of each stand at the end of the simulation are not allowed to be lower than a predefined standing volume limit under standard treatment circumstances. The optimization model additionally enables definition of annual minimum harvesting volumes. As we could not find any suitable optimization algorithm, we developed a novel model basing on the ?Simulated Annealing? method. Our software is specifically parameterizable to the very complex data structure of the TreeGrOSS software. Our optimization software is already freely available as a package for the statistical software R via the Comprehensive R Archive Network. The forest stand data must be stored in a PostgreSQL database. Nagel and Hansen (NW-FVA) developed a software able to translate forest inventory data into TreeGrOSS stands.
We present the core-element
\textbf{Chapter \ref{chap:opt}} is probably most interesting chapter for many readers.
\textsc{\begin{figure}
		\center
		\resizebox{0.7\linewidth}{!}{% Define block styles
\tikzstyle{decision} = [diamond, draw, 
text width=4.5em, text badly centered, node distance=3cm, inner sep=0pt]
\tikzstyle{block} = [rectangle, draw, 
text width=6em, text centered, rounded corners, minimum height=4em]
\tikzstyle{line} = [draw, -latex']
\tikzstyle{cloud} = [draw, ellipse, node distance=3cm,
minimum height=2em]
\tikzstyle{db} = [cylinder, draw, shape border rotate=90, minimum height=4em, text width=6em, text centered,
aspect=0.25]


\begin{tikzpicture}[node distance = 3.5cm, auto]
    % Place nodes
\node [block] (dev) {device};
\node [block, left of=dev] (sim) {growth simulation};
\node [block, right of=dev] (opt) {optimizer};
\node [db, below of=sim, node distance=2.5cm] (db) {database};
    
% Draw edges
\path [line] (sim) -- (db);
\path [line] (db) -- (sim);
\path [line] (sim) -- (dev);
\path [line] (dev) -- (sim);
\path [line] (dev) -- (opt);
\path [line] (opt) -- (dev);

\end{tikzpicture}}
		\caption{Example flowchart.}
		\label{fig:Introduction:flowopt}
	\end{figure}
}