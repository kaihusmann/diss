\setchapterpreamble[uc][.75\textwidth]{%
	\dictum[Davis, Johnson, Howard and Bettinger, \textit{Forest Management}]{%
		``Forest management, whether for timber production, biodiversity, or any other goals, requires decisions that are based on both our knowledge of the world and human values.''}\vskip1em}

\chapter{Introduction}
\label{chap:Introduction}
Decision making is the last step in the process of planning, starting with actually discovering existence of a decision problem with at least two distinct alternatives. Complexity of the planning process, thereby determined through the type of decision, may vary from very simple daily decisions to extensive and long-lasting processes \citep[p. 3-4]{kangas_2015}. Those two extremes differ fundamentally in terms of problem structures, consequences, preferences of the decision-maker and solution evaluation \citep[p. 807-808]{keeney_1982}. While everyday choices in professional framework usually base on associative selections and personal preferences, crucial decisions or decisions with long-lasting impact are often made analytically with explicit inference (\citealp[p. 659, 672]{stanovich_2000}; \citealp[p. 3]{kangas_2015}). Following classical nomenclature, there are two strategies of examining decisions scientifically. Decisions can be described descriptively or normatively \citep[p. 6]{bitz_2005}. Examining decisions from descriptive perspective means evaluating individual and social actions. Descriptive decision theory analyses realistic decisions with aim of examining how decision-makers act in reality and how decision making actually works. It analyses the principles of decision making descriptively without further investigation of purposes \citep[p. 499-501]{simon_1979}. Findings from descriptive studies hence do not allow drawing any conclusions about plausibility or reasonability. It is not necessarily said that lessons learned from empiric-descriptive analysis of decisions lack reasonability. Descriptive decision theory just not inquiries rationals behind decisions \citep[p. 500]{simon_1979}. Task of the normative decision theory is examination of particular reasons behind actual decisions. Hence, in contrast to the descriptive theory, normative studies are rather evidence-based than empirically driven. They aim on theoretical explanation of the causal network which leads to decisions. Normative decision models are, usually computer-aided, mathematical, statistical or numerical computations trying to explain decision processes accounting for their intrinsic criteria. Whenever decision-makers are interested in the reasons of fairly complex decision problems, normatively examination will be obligatory. Normative decision examination builds the basement of operations research \citep[p. 498]{simon_1979} an interdisciplinary discipline with elements from statistics, mathematics, economics and computer sciences which developed simultaneously with the first digital computers \citep{churchman_1957}. Operations research, also called management or decision sciences, is the science of building and using computer-aided models for decision support \citep[p. 373-374]{wacker_1998}. After successful identification of decision-relevant aspects, the gathered information can be used to built a statistical computer-aided model. A completed normative decision analysis is hence an obligatory step in decision support systems (DSS) development. Modern decision support models are lastly always linked with classical decision theory as they basically translate normatively inquiry into applicable models for scientists, practitioners or any person being responsible for decisions.




how can this help decision-makers in practical issues of forestry ... 

describing decisions in forestry

understanding sense and purpose of decisions helps

Decisions in forest management never affect things solely. Once made, management decisions will have impact on lots of economic, ecologic and social issues.


Nachhaltigketit
-> Es ist schwer in der FW Ents. zu treffen
Wir zeigen, wie Entscheidungen untertst�tzt werden k�nne. Insbesondere wichtig weil anford an wald steigen
K�nnen menshcl. Entscheid. nicht ersetzen aber unterst�tzen
Since \citet{Carlowitz_1713} introduced the principle of sustainability to forestry, it plays a central role in there, and over the last centuries it has been further developed and extended. To achieve and maintain sustainability in its different specifications \citep{Speidel_1984, Schanz_1996} can be seen as one of the main goals or even the main goal of forestry. A prerequisite for such a sustainable forestry is information on the forest resources, their conditions and changes. This information is usually gained through forest inventories.

\section{The role of bio-economy in foresty}
Bio-economy is defined as
which is not novel in forestry
it means novel industries
meaning today
outlook
Forschungsfrage: Mittel, die gestiegenen Anforderungen an den Wald zu bewaeltigen. Wie koennen die Potenziale des Waldes ausgeschoepft werden, ohne die Nachhaltigkeit zu gefaehrden?
\textsc{\begin{figure}
	\center
	\resizebox{0.7\linewidth}{!}{% Define block styles
\tikzstyle{decision} = [diamond, draw, 
text width=4.5em, text badly centered, node distance=3cm, inner sep=0pt]
\tikzstyle{block} = [rectangle, draw, 
text width=6em, text centered, rounded corners, minimum height=4em]
\tikzstyle{line} = [draw, -latex']
\tikzstyle{cloud} = [draw, ellipse, node distance=3cm,
minimum height=2em]
\tikzstyle{db} = [cylinder, draw, shape border rotate=90, minimum height=4em, text width=6em, text centered,
aspect=0.25]


\begin{tikzpicture}[node distance = 3.5cm, auto]
    % Place nodes
\node [block] (dev) {device};
\node [block, left of=dev] (sim) {growth simulation};
\node [block, right of=dev] (opt) {optimizer};
\node [db, below of=sim, node distance=2.5cm] (db) {database};
    
% Draw edges
\path [line] (sim) -- (db);
\path [line] (db) -- (sim);
\path [line] (sim) -- (dev);
\path [line] (dev) -- (sim);
\path [line] (dev) -- (opt);
\path [line] (opt) -- (dev);

\end{tikzpicture}}
	\caption{Example flowchart.}
	\label{fig:Introduction:flowopt}
\end{figure}
}

Reference example to chapter \ref{chap:hzb}.
