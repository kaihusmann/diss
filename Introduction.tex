\setchapterpreamble[uc][0.8\textwidth]{%
	\dictum[Davis, Johnson, Bettinger, Howard, \textit{Forest Management}]{%
		``Forest management, whether for timber production, biodiversity, or any other goals, requires decisions that are based on both our knowledge of the world and human values.''}\vskip1em}

\chapter{Introduction}
\label{chap:intro}
\newpage
\section{Theoretical background of decision making}
\label{sec:intro:bg}
Decision making is the last step in the process of planning, which starts with actually discovering the existence of a decision problem. The complexity of the planning process, is thereby determined by the type of decision and may vary from very simple daily decisions to extensive and long- lasting decision processes \citep[p. 3-4]{kangas_2015}. Relatively easy and quick decisions differ fundamentally from more complex decisions in terms of problem structure, consequences, preferences of the decision-maker and solution evaluations \citep[p. 807-808]{keeney_1982}. While everyday choices in a professional framework are usually based on associative selections and personal preferences, crucial decisions or decisions with long-lasting consequences are often taken analytically with explicit inference methods (\citealp[p. 659, 672]{stanovich_2000}; \citealp[p. 3]{kangas_2015}). These two different decision types, often called \textit{systems} in behavioral sciences \citep[e.g.][p. 658]{stanovich_2000}, underlie significantly different theorems. Following classical theory of decision examination, decisions can be described either \textit{descriptively} or \textit{normatively} \citep[p. 6]{bitz_2005}. Examining decisions from a descriptive perspective means the evaluation of individual and social actions. The descriptive decision theory analyses actual decisions with the aim of examining how decision-makers act in reality and how decision making actually works. Hence it analyses the principles descriptively without further investigation of the underlying purposes. Descriptive decision studies try to answer the question \textit{how} but not \textit{why} decision-makers decide \citep[p. 499-501]{simon_1979}. Findings from descriptive studies, therefore, do not allow the drawing of conclusions about plausibility or reasonability. Lessons from empiric-descriptive analysis, though, do not necessarily lack reasonability. Descriptive decision theory does not inquire into the rationale behind decisions \citep[p. 500]{simon_1979}. The objective of the normative decision theory is the examination of particular reasons behind decisions. Hence, in contrast to the descriptive theory, normative studies are evidence-based rather than descriptive. In particular, they aim to theoretically explain the causal network that leads to decisions. Normative decision models are usually computer-aided, mathematical, statistical or numerical computations trying to explain decision processes by accounting for their intrinsic criteria. If researchers or decision-makers are interested in the rationale behind fairly complex decision problems, a normative examination will usually be obligatory.

Normative decision examination constitutes the foundation of \textit{operations research} (\citealp[p. 112]{shim_2002}; \citealp[p. 498]{simon_1979}), an interdisciplinary science with elements from statistics, mathematics, economics and computer sciences, which developed simultaneously with the first digital computers \citep{churchman_1957}. Operations research, also known as management sciences or decision sciences, is the science of building and using computer-aided models for decision support \citep[p. 373-374]{wacker_1998}. Methods from operation research are nowadays mandatory tools for almost all crucial intermediate- and long-term decisions to be made in a professional framework. Operations research hence builds the theoretical background of all modern computer-aided programs for decision support, also called \textit{decision support systems} (DSS). The main aim of DSS is to support complex decision problems by providing crucial theory or inference-based informations to decision makers. In one of the most popular definitions of DSS, in \citet[p. 26]{gorry_1971}, it is stated that decision processes must be simplified and abstracted in a way that they can be programed as computer code. Type and intensity of the abstraction depend on the complexity of the decision problem. In this context, abstraction means the identification and simplification of relevant elements of a decision. Those elements can be theoretical or inference-based. A very simple and often used example to examine elements inference-based is the linear optimization. As one property of linear optimization is the assumption of linearity between decision variables and decision objective, application of linear optimization requires very strict simplifications \citep[p. 129]{kangas_2015}. A linear regression analysis is one possibility to force all crucial decision elements into a linear framework, thereby fulfilling the property of linear optimization. Prior to implementation of a linear optimization, as an example for a DSS, a linear regression analysis can hence be used as a tool to simplify the decision elements. The abstraction and simplification of decision processes into \textit{programmable} elements is usually performed using statistical inference, such as regression or variance analysis. Applied statistical models are therefore mandatory tools right from the beginning of operations research \citep{churchman_1957}. All further decision elements, in particular elements that are unstructured or far too complex to simplify them with statistical methods, are called \textit{non-programmable} decision elements. Those elements cannot be considered in a DSS. Prior to implementation into computer-aided models, decision-relevant aspects must be gathered, reviewed and simplified. A complete and accurate normative decision analysis is, therefore, a prerequisite for the development of a DSS. Almost all modern decision support models are based on classical decision theory, as they basically translate normative inquiry into applicable models for scientists, practitioners or any other decision maker. Next to the actual variables and rules, typical DSS also have a user front-end and a data-warehouse (\citealp[p. 2]{hansen_2012}; \citealp[p. 115]{shim_2002}). The user front-end facilitates the application of a DSS for the user. The data-warehouse enables storage of necessary input data and the solution.

\section{Decision support systems in forest planning}
\label{sec:intro:dss}
Decisions in forest management never affect particular issues in isolation. Once made, management decisions in forestry will impact on many economic, ecological and social issues. A challenge foresters typically have to cope with is the long-term consequence of their operations. Daily operational decisions of foresters, such as harvesting at the stand level or planting, usually have very long lasting consequences. Foresters must, therefore, review the consequences of their decisions very thoroughly. Forests should be managed in such a way that they produce income for the forest owner on the one hand, while at the same time following conservation and recreational issues on the other hand (\citealp[p. 11]{kangas_2015}; \citealp[p. 67]{mohring_1997}). Simultaneous fulfillment of all three functions, the ecological, economic and social functions - on the entire forest land is the guiding principle of multifunctional forestry in Germany. It is firmly anchored in the Federal Forest Law in Germany \citep[p. 457]{moller_2007}. The concept of \textit{sustainability} plays a central role in forestry. In one of its most recent and general definitions, sustainability means the development of forests such that current and future generations can benefit from all three forest functions (\citealp[p. 14]{un_2005}; see also \citealp[p. 14]{kangas_2015}). As a consequence, forest management decisions may not lead to a decline in several aspects, such as \textit{biodiversity}, \textit{productivity} and \textit{regeneration capacity vitality} (Ministerial Conference on the Protection of Forests in Europe, cited from \citealp[p. 15]{kangas_2015}). Forest management thus requires careful planning, considering multiple criteria at the same time. Owing to this high complexity, crucial decisions in practical forestry are rarely made by single persons. Operational forest decisions are usually based on a complex synthesis of intermediate-term plans and long-term strategies that enable long-term issues, such as general forest development or nature conservation issues, to be taken into account in practical daily operations. Daily operational forest decisions made by foresters, e.g. harvesting or planting on stand level, normally rely on intermediate-term management guidelines that are created for periods between 5 and 20 years \citep[p. 12]{kangas_2015}. The German intermediate-term forest management planning is, especially in the public forests, a long-established, continuously improved process which tries to implement a strategic orientation into spatially and temporally explicit guidelines for operational decision makers \citep[p. 156-158]{bockmann_2004}. The entire forest planning process is, hence, a complex framework that is comprised of long-, intermediate- and short-term management plans with the aim of supporting the operational decisions of foresters with respect to a wide range of relevant issues.

Next to an increasing demand for wood as raw materials \citep[p. 8]{mantau_2012}, requirements on conservation and recreational issues are rising as well. For example, Germany's national strategy for biological diversity has the goal of natural development on 10 \% of public owned and 5 \% of private owned forest land \citep[p. 45]{bmu_2007}. As a consequence, more than 700,000 hectares of forest land are planned to be set-aside by 2020 \citep{ti_2014}. \citet[p. 3]{auer_2016} calculated the wood potential, particularly for European beech, in the center of Germany. They concluded that the potential was already almost completely exhausted in the period between 2002 and 2012. An ongoing increase in wood demand coupled with a decrease in the available productive area and a simultaneous increase in recreational issues \citep[p. 1]{hansen_2012}, while still maintaining positive returns, poses new challenges for forestry and the entire wood sector. The degree of complexity in forest planning is thus expected to increase even further.

This development explains the need for detailed, rational decision support \citep[p. 2]{hansen_2014}. If demand further rises, innovative computer-aided supply chains, as well as processing schemes, become even more mandatory in order to serve all wood consumers properly. DSS can be useful tools to simplify the complex framework of forestry decisions. They help by structuring the highly complex forest decision problems into smaller, solvable sub-problems. The advantages of DSS for forestry purposes were already discovered in the early 1980s \citep[p. 499]{reynolds_2008}. Nowadays, numerous examples of useful DSS can be found in forest practice and forest sciences covering a very broad range of purposes. Some very simple forest DSS are, for example, the \textit{generalized maximin method} or the \textit{certainty equivalent method} \citep[p. 25, 28]{kangas_2015}. Complex inference based decision support tools like the WaldPlaner \citep{hansen_2014} have already become relevant in the practical intermediate-term forest planning, e.g. in Lower Saxony \citep[p. 158]{bockmann_2004}. \textit{WaldPlaner} is a DSS for practitioners and scientists with a user friendly interface that combines a data-warehouse with the long-established and widely used tree growth and yield simulation software \textit{Tree Growth Open Source Software} (TreeGrOSS) (\citealp[p. 46]{hansen_2014}; \citealp{nagel_2009}). It enables, for instance, growth and yield simulation of multiple forest stands. It offers foresters the opportunity to review the consequences of management decisions and hence supports sophisticated evaluation of forestry decisions. Computer-aided forest simulation software in general have great potential to improve the effectiveness and accuracy of forest planning \citep[p. 210]{davis_2001}. They are superior to classical yield-tables since they can consider many more relevant aspects for growth and yield, such as species mixtures, complex within-stand structures and competition, as well as specific growth and treatment  rules (\citealp[p. 3]{hansen_2012} \citealp[p. 93]{muys_2010}).

\section{Optimization of forest planning}
\label{sec:intro:opt}
Right from the beginning of the use of DSS in forestry, optimization techniques came into use \citep[p. 16]{kangas_2015}. As optimization procedures were initially developed for efficient allocation of finite resources \citep[p. 271]{davis_2001}, they are suited to many decision problems in forestry, a field with naturally scarce resources. Combinations of modern growth and yield simulation software and optimization procedures are currently the focus of forestry research. A simultaneous application of growth simulators and optimization procedures offers opportunities for forest planning, as it enables foresters to consider environmental circumstances which change over a longer timeframe (such as climate change or nitrogen deposition) in their operational short-term decisions (\citealp[p. 346-347]{mohring_2010}; \citealp[p. 1081]{pretzsch_2008}). A recent example from forest sciences can be found in \citet{yousefpour_2009}. They developed an approach to optimize thinning operations in terms of economic return, which also takes timber production, carbon storage and biodiversity constraints into account. They combine the growth and yield simulation software TreeGrOSS with the \textit{dynamic linear programming} optimization procedure. They used an integrated simulation-optimization approach to estimate the monetary drawback of different treatment and nature conservation scenarios. Combined simulation-optimization methods are not novel in forestry. The United States Forest Service, for example, has applied such combined methods in practical forest planning since the 1980s \citep[p. 33]{hoganson_2015}. Other countries also have experience in the practical application of combined methods. Recently combined methods have found practical use in, for instance, the USA and Finland \citep[p. 41]{hoganson_2015}. Since both growth and yield simulation models, as well as optimization methods, have developed considerably in the last decade, modern combined methods have the potential to solve more sophisticated decision problems in forestry (\citealp[p. 16-17]{kangas_2015}; \citealp[p. 93]{muys_2010}).

\section{The role of bio-economy in forestry}
\label{sec:intro:biecon}
Bio-economy is defined as a combination of all economic sectors that refine biological resources by physical, chemical and biotechnological processes \citep[p. 10462]{debesi_2015}. The bio-economy itself is no novel sector but an aggregation of formerly separately regarded sectors, which are all based on biomass as a major resource. The main advantages of bio-economy as an aggregated sector are the research cooperation of formerly distinct companies in order to benefit from synergy \citep[p. 1]{auer_2016}, as well as commonly evaluated supply chains and resource demands \citep[p. 3]{geldermann_2016}.

With an overall turnover of 2 trillion {\euro} in 2014, including resources from agriculture, forestry and fishery, the European Union's bio-economy sector leads in a worldwide comparison \citep[p. 221, 223]{elchichakli_2016}, with Germany playing a particularly important role \citep[p. 200]{hennig_2016}. Although the share of forestry itself as a primary producer only amounts to 2 \% (35 billion {\euro}), forestry currently plays a crucial role for resource supply in the bio- economy sector. The bio-based sector is a factor of considerable importance in Germany's national economy as well. The bio-based economy, including primary production as well as manufacturing and services, accounted for approximately 8 \% of Germany's gross value-added in 2007 \citep[p. 29-30]{efken_2012}. Innovative bio-based products have potential as substitutes for end- and semi-finished products that are traditionally based on fossil resources. In the national bio-economy strategy, the German Government therefore decided to strengthen the bio-based sector until 2030 \citep[p. 15-16]{bmel_2014b}. Novel production methods could enhance the significance of forest biomass (in particular of small dimensioned wood) for use in bio-refineries. \citet[p. 49]{ekman_2013} revealed woody biomass to have great potential for chemical semi-finished products. Regarding the political and economic circumstances, the importance of the bio-economy sector in general, and in particular of forestry, is thus expected to increase further. The demand for wood is high and steadily rising. From a forestry perspective, the question arises, whether prospective demands for woody biomass can still be served with the available resources.

Cooperation between wood processing companies provides advantages, particularly for forest enterprises, with respect to improving planning possibilities. If the supply chains of formerly distinct smaller companies are joined together to an integrated super-regional supply chain which includes interactions between companies, logistical planning for forest enterprises will be considerably simplified \citep[p. 3]{geldermann_2016}. Supplying to well-prepared, centrally controlled networks can be beneficial in terms of increasing planning reliability and reducing planning costs, as communication between decision makers in bio-economy and forestry will be structured and therefore facilitated.

Modern modeling techniques already enable reliable forecasting of wood potential from forests. With help of DSS, forecasting of the expected wood potential is already frequently applied in the context of cluster studies \citep[e.g.][]{bmel_2016}, scientific studies (e.g. chapter \ref{chap:hzb}) and intermediate-term forest planning \citep[e.g.][]{bockmann_2004}. Wood supply can already be forecast reliably for time horizons up to 20 years. Sound knowledge of the respective wood demand will therefore have a considerable positive effect on intermediate-term forest planning. For similar reasons bio-economy DSS can provide useful tools for the assessment of wood demands. One aim of the collaborative bio-economy sector is the development of such super-regional and interactive DSS \citep[p. 362]{ollikainen_2014}. From the perspective of forestry, the collaboration of wood processing companies is desirable, since it will increase the reliability of wood demand studies. A common forecast of wood demand from a network of joined companies offers advantages, as it enables reliable matching of the forecast wood potential with prospective demand. Valid information on the required resources of wood processing networks can improve planning security, benefiting both bio-economy companies as well as forest enterprises.

Innovative bio-economy companies, such as bio-refineries, need a continuous wood supply to ensure ongoing manufacturing process \citep[p. 362]{ollikainen_2014}. Their success crucially depends on delivery contracts which ensure continuous wood supply. This meets the requirements of the forest sector, as delivery contracts can be beneficial for forest enterprises as well. Delivery contracts facilitate intermediate-term planning for both sides. Contractually determined continuous wood supply, on the other hand, leads to a limitation of the forest treatment possibilities. It restricts the possible harvesting operations. Wood usage ahead of the standard treatment schedule, as it is sometimes necessary to fulfill contractually determined minimum wood delivery amounts, can lead to usages exceeding growth in specific forest stands. To meet the concept of sustainability, however, each utilization above the growth must necessarily lead to reduced utilization at another time point \citep[p. 67]{mohring_1997}. The forester can thus be forced to harvest stands at unfavorable time points. This reduces the options of stand treatment within a forest enterprise \citep[p. 351-352]{mohring_2010}. As introduced in section \ref{sec:intro:dss}, the intermediate-term action plan represents the most favorable forest development schedule under a given strategic orientation of the forest enterprise. If all guidelines are respected properly, the standard treatment therefore represents the maximal harvestable wood volume without violating the intrinsic strategy of the forest enterprise. Any distortion will lead to a deviation from the preferred forest development. The difference in harvestable wood potential between the unrestricted treatment and the treatment under delivery restriction can be interpreted as the opportunity cost of the delivery contract \citep[p. 353]{mohring_2010}. It is thus the price that a forest enterprise has to pay for the benefits of delivery contracts.

\section{Aim of the thesis}
\label{sec:intro:aim}
Biomass from forests has the potential to provide a largely carbon-neutral supply of material to the bio-based sector and could therefore make a significant contribution to a clean bio-based industry. Modern utilization techniques enable the substitution of fossil resources by renewable biological resources. In this context, the bio-economy contributes to reducing the dependency upon fossil raw material and thus to the reduction of carbon dioxide emissions \citep[p. 4]{ingrao_2016}. Dedicated political programs and comprehensive research projects have strengthened the development of the bio-economy sector worldwide and show its current and prospective importance. The increasing political, social and economic importance of the bio-economy \citep[e.g.][p. 15-16]{bmel_2014b} reveals a worldwide process of rethinking towards a cleaner production.

The forest sector, as an important primary producer of renewable resources for the bio-economy, plays an important role for the success of the bio-economy. The rising demand for wood, however, could exceed the sustainable achievable wood potential of specific assortments. Reasonable distribution of the scarce resources is, therefore, a major challenge, which the forest sector has to face. In times of simultaneously rising demands on all three forest functions, the main challenge is \textit{''How to match the resource demands of a rising bio-economy sector with the available wood potential without compromising the concept of sustainability?''}

I present differentiated applied statistical analyses which strengthen distinct decisions in the wood supply chain of the bio-economy. I will introduce a descriptive analysis to calculate an overview of available wood potential in an important supply region and distinct normative inquiries investigating the decision elements of three relevant decision problems for supply of the bio-economy sector.
	
\section{Structure of the thesis}
\label{sec:intro:struct}
After the general introduction, I present distinct essays which deal with the research questions and problems mentioned in the introduction. Chapter \ref{chap:discussion} completes the thesis with a general discussion of all four studies.

\subsection{Analyzing status and development of raw wood availability in the European beech-dominated central Germany}
\label{subsec:intro:struct:hzb}
The initial step of a planning process is the actual identification of a decision problem (section \ref{sec:intro:bg}). Prior to calculation of methods to overcome resource distribution problems, the existence of a resource scarcity must be examined. If resources are not scarce, there will be no resource distribution problem. Availability above a particular demand will result in an oversupply. Supplying all market participants with desired raw material would be trivial in such a scenario. The first reasonable step to tackle a resource distribution problem is hence an estimation of the actual availability and demand of the respective resource.

The first approach to answer the research question is a cluster study to analyze the available wood potential and the market situation in a possible wood supply region for the bio-economy in Germany. The current and future availability of European beech (\textit{Fagus sylvatica} [L.]) raw wood, one of the most important wood resources for the growing bio-economy sector \citep[p. 16]{auer_2016}, was investigated in the beech-dominated center of Germany. Chapter \ref{chap:hzb} shows the results of a study analyzing the German federal states of Lower Saxony, North Rhine-Westphalia, Hesse, Saxony- Anhalt and Thuringia in terms of their beech wood potentials and demands.

The potential raw material within this supply region was calculated using the publicly available database of the German National Forest Inventory. These data were advantageous, since they represent a high-resolution systematic grid of sample points over the whole supply region and consider all ownership types \citep{schmitz_2008}. The future wood potential was then forecast using the forest DSS WaldPlaner \citep{hansen_2014}.

The study is an example of how inventory data extrapolation and forest simulation methods can be applied to create a quantitative base to support the strategic orientation of the bio-economy sector. Profound wood potential analysis on current and predicted wood amounts can provide valuable information for upcoming and established companies and help in appraising raw material availability for prospective production.

\subsection{Biomass functions and nutrient contents of European beech, oak, sycamore maple and ash and their meaning for the biomass supply chain}
\label{subsec:intro:struct:bm}
Modern utilization techniques in the fields of bio-economy are able to make use of smaller dimensioned wood of, in particular, broadleaf species. Because the chemical constituents of the wood are dissolved in innovative bio-refineries, novel bio-economy companies are mainly interested in the dry woody biomass, rather than in the dimension or form of the wood assortments (\citep{ekman_2013}). These novel companies are thus interesting from the point of view of forest enterprises, as the smaller wood residuals left after harvesting the stem wood presently often remain in the forest. The usage of small dimensioned branches is, however, controversial. Too high nutrient exports lead to soil degradation and are therefore not compatible with the concept of sustainability \citep[p. 261]{pretzsch_2014}. Biomass functions and nutrient contents are useful tools for estimating the acceptable degree of harvesting intensity.

Biomass functions have two advantages for the bio-based sector. They can help in exploiting the available wood potential, in particular for small wood and they can support decisions on the raw material supply chain. They enable estimation of single tree dry biomass via easily measurable tree attributes and therefore allow valuation of the biomass flow in the supply chain. The potential of forest sites can only be entirely exploited when the avoidance of soil exhaustion is also taken into account. For this, the estimation accuracy of the site-specific wood potential can be improved by using realistic biomass functions and nutrient contents. Biomass functions and nutrient contents are available for European beech and oak (\textit{Quercus robur} [L.] and \textit{Quercus petraea} [Matt.]) but not for sycamore maple (\textit{Acer pseudoplatanus} [L.]) and ash (\textit{Fraxinus excelsior} [L.], which often occur in mixture with beech \citep{ti_2014}. Accurate biomass functions and nutrient contents for these species can thus help to unlock additional wood potential for the bio-economy that was, due to the lack of a basis for the calculation of compatible harvesting intensity, so far unused.

In chapter chapter \ref{chap:bm}, biomass functions and nutrient contents for European beech, oak, sycamore maple and ash are introduced. Their meanings for the supply of a bio-based industry with woody biomass are discussed in detail. The biomass and nutrient content models can then be used for the implementation into DSS, such as the WaldPlaner \citep{hansen_2014}, in order to enable quick calculations of the site-specific wood potential and to calculate material flows of woody biomass.

It is shown how inventory methods for natural resources can be used to efficiently estimate the biomass of single trees. Generalized linear and nonlinear regressions were used to calculate biomass functions and nutrient contents for beech, oak, sycamore maple and ash. How multicollinearity problems in biomass measurements influence the estimate and the variance of nonlinear biomass models is empirically determined and discussed.

\subsection{Modelling the economically viable wood in the crown of European beech trees}
\label{subsec:intro:struct:beech_crowns}
Even if the site specific ideal harvesting intensity is acknowledged to be valid, the economic viability of this wood potential may not be given. In chapter \ref{chap:beech_crowns}, a program to predict the economically viable wood of European beech crowns is presented.

The model, which is able to distinguish economically viable from unviable branches in the crowns of European beech trees, was programmed to calculate the maximal single-tree wood potential with respect to economic objectives. It therefore helps to enable full exploitation of the timber potential on the single tree level. It is the first model in scientific forestry literature that predicts the wood volume in European beech crowns with respect to the complex sympodial crown structure. It has advantages over available models, as it is not based on taper models but on actual morphological measurements. The model was performed on 163 European beech trees to calculate their individual economically viable wood volume. By performing regression analysis, the model results were used to develop a regression formula able to predict the economically viable wood volume in the crown.

The regression formulas were developed such that they can easily be implemented into DSS. They offer opportunities for the decision makers to assess the full wood potential from an economic perspective. The evidence-based prediction of the full tree-specific wood potential has two main advantages. The prediction of the actual harvestable wood volume is facilitated by the models. Prior to harvesting, foresters can use the model to easily estimate the processing intensity for optimal monetary return. They can thus assess the full wood potential of every harvesting operation. This enables the gathering of formerly unused wood volume and hence increases the wood potential for the bio-economy sector. The predictive model promises further advantages for operational planning. As accuracy of predicting the harvestable wood is enhanced, the viability of entire wood volume or biomass supply chains can be strengthened.

The estimation model of viable wood volume in European beech trees is basically an integration of a break-even analysis into the predictor of the multistage randomized branch sampling method. It is a combination of a biometric sampling strategy for single tree attributes \citep[p. 405]{Gregoire_2008} with an econometric critical value analysis \citep[p. 46]{mushoff_2013}. Generalized linear and nonlinear regression, cluster analysis and linear discriminant analysis are used to parameterize applicable formulas for implementation into DSS.

\subsection{Flexible Global Optimization with Simulated-Annealing}
\label{subsec:intro:struct:opt}
Although optimization procedures are already important planning tools in an international framework \citep[p. 1]{hoganson_2015} and have shown their potential to support operational planning, while simultaneously considering long-term issues of (\citealp[p. 1]{hoganson_2015}; \citealp[p. 1081]{pretzsch_2008}), they have so far only played a minor role in Germany. Steadily increasing demands for wood, as well as for further ecosystem benefits, makes forest harvesting planning in Germany increasingly more complex (section \ref{sec:intro:dss}). For this reason, combined simulation-optimization DSS could be a promising tool for German foresters and forest scientists to support harvesting operations decisions in terms of intensity and time. Having a closer look at optimization methods seems, therefore, to be worthwhile. In chapter \ref{chap:opt}, a combined simulation-optimization DSS for support of intermediate-term forest harvesting planning is introduced, which is specifically adopted to German characteristics. Optimization of forest growth and yield is very complex and therefore makes high demand on the optimization procedure. The essay preliminary deals with the opportunities and limitations of different optimization procedures for use in forestry decision support. Finally, I present an optimization procedure, which is able to tackle the complex output of forest simulation software. An explicit example is then developed. The software enables calculation of the optimal thinning intensity in time horizons of up to 20 years, and takes sustainability and the strategic orientation of a forest enterprise into account.

The presented optimization procedure is part of a simulation-optimization software, which is currently comprised of four basic elements (Figure \ref{fig:Introduction:flowopt}). The first element is the growth simulation, which performs the actual growth and yield simulations. Tree growth and yield are simulated using TreeGrOSS, a long established single-tree based simulation software of the Northwest German Research Institute. TreeGroSS is also the back-end of the widely used forest DSS WaldPlaner \citep[p. 6-7]{hansen_2014}. The growth and yield simulation element is a stand-alone Java written software, developed by \citet{nagel_1996}. The TreeGrOSS packages, formerly known as NEWS, are advantageous for an integrated simulation-combination system, since they are one of the oldest and most often used growth and yield software in Germany. TreeGrOSS is compatible with a variety of data-bases \citep[p. 55]{hansen_2014} that can serve as the data-warehouse for storing the raw data and the results of the simulations.
\textsc{\begin{figure}[h]
		\center
		\resizebox{0.7\linewidth}{!}{% Define block styles
\tikzstyle{decision} = [diamond, draw, 
text width=5.5em, text badly centered, node distance=3cm, inner sep=0pt]
\tikzstyle{block} = [rectangle, draw, 
text width=8em, text centered, rounded corners, minimum height=6em]
\tikzstyle{line} = [draw, -latex']
\tikzstyle{cloud} = [draw, ellipse, node distance=10cm,
minimum height=2em]
\tikzstyle{db} = [cylinder, draw, shape border rotate=90, minimum height=6em, text width=8em, text centered,
aspect=0.25]


\begin{tikzpicture}[node distance = 4cm, auto]
    % Place nodes
\node [block] (dev) {interface};
\node [block, left of=dev] (sim) {growth and yield simulation};
\node [block, right of=dev] (opt) {optimizer};
\node [db, below of=sim, node distance=3.3cm] (db) {data-\\warehouse};
    
% Draw edges
\path [line] (sim) -- (db);
\path [line] (db) -- (sim);
\path [line] (sim) -- (dev);
\path [line] (dev) -- (sim);
\path [line] (dev) -- (opt);
\path [line] (opt) -- (dev);

\end{tikzpicture}}
		\caption{The four basic elements of the combined simulation-optimization software.}
		\label{fig:Introduction:flowopt}
	\end{figure}
}

The interface (Figure \ref{fig:Introduction:flowopt}) links the simulation with the optimization module. It has the job of translating the TreeGrOSS in- and output into state and parameter spaces that are interpretable by the optimization software. The interface is an R written function that internally calls the growth and yield modules of TreeGrOSS. The function can be called in any R session and thus allows TreeGrOSS based growth and yield simulations to be run directly from R. \textit{R} is a flexible statistical programing language allowing relatively easy implementation and manipulation of optimization procedures and additional features\citep[p. 11-12]{nash_2014}. The interface thus enables easy connection between TreeGrOSS and many optimization libraries.

In TreeGroSS, harvesting intensity is specified by the difference between actual and user definable target basal area \citep[p. 149-150]{hansen_2014}. The interface function translates all target basal areas of all forest stands $n_{stands}$ and all simulated years $n_{years}$ of an optimization problem into a matrix with numeric values. The resulting objective matrix, which can be passed to the interface function, is thus of dimension $n_{stand}$ x $n_{years}$, where $n_{years}$ can also describe discrete steps of more than one year. Within the interface function, the values from the objective matrix are translated into TreeGrOSS interpretable goal basal areas. After translation, the interface function internally calls the growth and yield libraries basing on those entries. The resulting harvesting wood volumes from the TreeGrOSS simulation are stored and further processed in the interface function. The volumes are rated in terms of costs and revenues, summed and finally returned. The interface function is, in principle, a function that enables manipulation of the crucial TreeGrOSS simulation settings from R.

One of the most interesting and most challenging properties of forest planning optimization is the comprehensive and straightforward definition of the sustainability principle \citep[p. 15]{kangas_2015}. Simulation-optimization DSS enable the objectifying of the principle of sustainability as explicitly defined computer rules. To consider sustainability in the optimization algorithm, I implemented two restrictions in the interface. In the last simulated year, the total standing volume (the sum of the standing volumes in all stands) is not allowed to be lower than a predefined limit. Additionally, each distinct standing volume in the last simulated year must be above a distinct stand-specific minimum limit standing volume. The predefined limits are based on growth and yield simulations under standard treatment circumstances. The interface function simply returns no valid value ({\tt NA}), when a simulation result is outside the restriction limits. Besides the maximal harvesting volumes, which are determined by sustainability limitations, the interface also enables the definition of minimum harvesting volumes in the form of a further user selectable restriction. The optimization model hence enables definition of user defined annual minimum harvesting volumes for four assortments. The user can separately parameterize minimum annual harvesting volumes for deciduous and coniferous stem and industrial wood.

To conclude, the combined simulation-optimization introduced here, is based on sequential TreeGrOSS simulations with iteratively changing thinning intensity settings. Every iteration in the optimization progress is comprised of a TreeGrOSS simulation and rating of the harvested wood volume. As the TreeGrOSS packages are a complex causal network of rules, linear and nonlinear equations, the interface returns an irregular response pattern, including undefined parts. These properties make high demands on the optimization method. Many simpler functions are not suitable for the highly complex state space of the optimization problem. Extensive analyses of prospective optimization algorithms, including linear programming, direct-search and random-search heuristics, revealed the need for flexible heuristic optimization techniques to solve the problem. A very flexible optimization method without many assumptions on the loss function is presented and applied exemplary in chapter \ref{chap:opt}.

The combined model is an example of how applied statistical modeling can be used to strengthen the intermediate-term forest planning in times of a growing bio-economy sector. It offers an opportunity to optimize the intensity of harvest operations, thus enlarging harvestable wood potential for the bio-economy, without violating sustainability and enterprise-intrinsic strategic orientations. The combined simulation-optimization model provides a means to evaluate whether annual continuous delivery contracts are actual feasible without violating sustainability principles. With regard to the research question, the simulation-optimization model can be used to examine whether forest enterprises are able to comply with delivery contracts, keeping in mind their specific tree structures and intrinsic strategies. It will thus enable a sophisticated analysis of the ability of a forest enterprise to match the demands of the bio-economy. 

It even allows further examination of the impact of delivery contracts. In a formal optimization framework, delivery contracts are restrictions that limit the possibilities of stand developments. Binding restrictions will necessarily have negative effects on the objective. This means, that the total harvestable volume in the time period of the optimization is either unaffected or decreased by delivery contracts with minimum annual wood amounts. Next to the benefits for forestry and bio-economy companies, delivery contracts also have opportunity costs for the forest enterprise (see also section \ref{sec:intro:biecon}). The simulation-optimization model makes calculating those opportunity costs possible by comparing the actual optimal treatment with the restricted optimum. It can, therefore, be a useful tool for assessing the advantages and drawbacks of continuous wood delivery rates from the perspective of forest enterprises. This information can be used to find an objective trade-off between the costs and the benefits of annual wood delivery quantity.