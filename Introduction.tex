\setchapterpreamble[uc][0.8\textwidth]{%
	\dictum[Davis, Johnson, Howard and Bettinger, \textit{Forest Management}]{%
		``Forest management, whether for timber production, biodiversity, or any other goals, requires decisions that are based on both our knowledge of the world and human values.''}\vskip1em}

\chapter{Introduction}
\label{chap:intro}
Decision making is the last step in the process of planning, starting with actually discovering the existence of a decision problem. Complexity of the planning process, thereby determined through the type of decision, may vary from very simple daily decisions to extensive and long-lasting decision processes \citep[p. 3-4]{kangas_2015}. Relatively easy and quick decisions differ fundamentally from complexer decisions in terms of their problem structures, consequences, preferences of the decision-maker and solution evaluation \citep[p. 807-808]{keeney_1982}. While everyday choices in professional framework usually base on associative selections and personal preferences, crucial decisions or decisions with long-lasting impacts are often taken analytically with explicit inference (\citealp[p. 659, 672]{stanovich_2000}; \citealp[p. 3]{kangas_2015}). Those two different typos, often called \textit{systems}, in behavioral sciences \citep[e.g.][p. 658]{stanovich_2000}, underly significantly different theorems. Following classical theory of decision examination, decisions can be described either \textit{descriptively} or \textit{normatively} \citep[p. 6]{bitz_2005}. Examining decisions from descriptive perspective means evaluating individual and social actions. Descriptive theory analyses actual decisions with aim of examining how decision-makers act in reality and how decision making actually works. It analyses the principles of decision making descriptively without further investigation of purposes. Descriptive decision studies will answer the question \textit{how} but not \textit{why} decision-makers decide \citep[p. 499-501]{simon_1979}. Findings from descriptive studies hence do not allow drawing any conclusions about plausibility or reasonability. Lessons from empiric-descriptive analysis, however, do not necessarily lack reasonability. Descriptive decision theory just not inquiries rationals behind decisions \citep[p. 500]{simon_1979}. Target of the normative decision theory is examination of particular reasons behind actual decisions. Hence, in contrast to the descriptive theory, normative studies are rather evidence-based than descriptive. They aim on theoretical explanation of the causal network which leads to decisions. Normative decision models are, usually computer-aided, mathematical, statistical or numerical computations trying to explain decision processes accounting for their intrinsic criteria. Whenever decision-makers are interested in the reasons of fairly complex decision problems, normatively examination will usually be obligatory. Normative decision examination forms the foundation of \textit{operations research} (\citealp[p. 112]{shim_2002}; \citealp[p. 498]{simon_1979}), an interdisciplinary discipline with elements from statistics, mathematics, economics and computer sciences which developed simultaneously with the first digital computers \citep{churchman_1957}. Operations research, also called management or decision sciences, is the science of building and using computer-aided models for decision support \citep[p. 373-374]{wacker_1998}, nowadays mandatory tools for intermediate- and long-term decisions in professional framework. Operations research hence builds the theoretical background of all modern computer-aided programs for decision support, also called \textit{decision support systems} (DSS). DSS are computer system of crucial variables and rules that structure the programmable part of a decision problem. Since the initial definition of DSS in \citet[p. 26]{gorry_1971}, it is stated that decision processes must be simplified into programmable elements. As programmable the elements of DSS are usually examined via statistical inference such as regression or variance analysis, applied statistical models are mandatory tools right from the beginning of operations research \citep{churchman_1957}. All further elements, in particular non-programmable processes that are unstructured or far too complex, cannot be considered in DSS. Prior to implementing statistical computer-aided models, decision-relevant aspects must be gathered, reviewed and simplified. Only after successful identification of programmable and meaningful decision-relevant aspects, the gathered information can be used to built a statistical computer-aided model. A complete, accurate normative decision analysis is hence a prerequisite for the development of DSS. Most all modern decision support models lastly base upon classical decision theory as they basically translate normatively inquiry into applicable models for scientists, practitioners or any person being responsible for decisions. DSS aim on objective, theory-based solutions of real decision problems. Next to the actual variables and rules which are translated into program code, typical DSS therefore have a user front-end and a data-warehouse (\citealp[p. 2]{hansen_2012}; \citealp[p. 115]{shim_2002}). The user front-end facilitates application for the user, the data-warehouse enables storage of necessary input data and the solution.

Decisions in forest management will never affect things solely. Once made, management decisions will have impact on lots of economic, ecologic and social issues. Forests should be manged such that they produce income for the forest owner \textcolor{red}{(CITE M�Hing. 1997 ODER Waldeigentum 923 Dep in BBF)} on the one, while contemporaneously follow conservation and recreational issues, on the other hand \citep[p. 11]{kangas_2015}. Simultaneous fulfillment of all 3 functions on the entire forest land is the guideline principle of the multifunctional forestry in Germany. It is firmly anchored in the Federal Forest Law \citep[p. 457]{moller_2007}. The concept of \textit{sustainability} plays a central role in forestry. In one of its most recent and general definitions sustainability states development of forests such that current and future generations can benefit from all 3 forest functions (\citealp[p. 14]{un_2005}; see also \citealp[p. 14]{kangas_2015}). In consequence, practice forest management decisions must exclude declining of several different aspects like \textit{biodiversity}, \textit{productivity} and \textit{regeneration capacity vitality} (Ministerial Conference on the Protection of Forests in Europe, cited from \citealp[p. 15]{kangas_2015}). Forest management thus requires careful planning considering multiple criteria in each decision. Due to this complexity, decisions in practical forestry are rarely made by single persons. Forest operations planning e.g. usually bases on a complex collaboration of inventory and action planning being comprised of long-term strategies, intermediate-term management plans and short-term operational decisions. Intermediate-term forest planing typically relies on the forest mensuration which is a long-established, continuously improved planning procedure trying implementing strategic orientations into spatial and temporal explicit operations \citep[p. 156-158]{bockmann_2004}. In addition to the high degree of complexity, a further challenge foresters typically have to cope with, is the long-lasting consequence of their operations. In forestry definition, intermediate-term planning already comprises time periods of 5-20 years \citep[p. 12]{kangas_2015}. Daily operational decisions of foresters, such as harvesting on stand level or planting will usually have consequences that last much longer than any plan. Decision maker thus must review consequences of their decisions thoroughly. Next to an increasing demand for wood as raw materials \citet[p. 8]{mantau_2012}, requirements on conservation and recreational issues are rising as well. Germany's national strategy of biological diversity e.g. provides natural development on 10\% of public owned and 5\% of private owned forest land \citep[p. 45]{bmu_2007}. More than 700,000 hectares of forest land are hence to be set-aside by 2020 \citep{ti_2014}. \citet[p. 3]{auer_2016} calculated particularly European beech wood potential in the center of Germany. They came to the conviction that the potential was almost completely used in the time period 2002 to 2012. Increasing demand for wood, decreasing available productive area and contemporaneously increasing recreational issues \citep[p. 1]{hansen_2012} while still producing positive returns pose new challenges for the forestry and wood sector. The degree of complexity in forest operations is thus expected to increase even further which explains the need for detailed, rational decision support \citep[p. 2]{hansen_2012}. If demand further rises, innovative computer-aided supply chains as well as processing schemes become mandatory to serve all wood consumers.

DSS can be used to structure the highly complex decision problems into smaller, programmable sub-problems and finally facilitate and solve many decision problems in forestry. The advantages of DSS for forestry purposes were already discovered in the early 1980s \citep[p. 499]{reynolds_2008}. Nowadays, numerous examples of useful DSS can be found in forest practice and forest sciences covering a very broad range of purposes. Very simple DSS are e.g the generalized \textit{maximin} method or the \textit{certainty equivalent} method \citep[p. 25, 28]{kangas_2015}. Complex inference based decision support tools like the \textit{WaldPlaner} \citep{hansen_2014} are already relevant part in the practical forest mensuration e.g. in Lower Saxony \citep[p. 158]{bockmann_2004}. WaldPlaner is a DSS for practitioners and scientists with user-friendly interface that combines a data-warehouse with the long-established and widely used tree growth and yield simulation software \textit{Tree Growth Open Source Software} (TreeGrOSS) (\citealp[p. 46]{hansen_2014}; \citealp{nagel_2009}). It enables i.e. growth and yield simulation of multiple forest stands. It offer foresters opportunity to review consequences of management decisions and hence supports sophisticated evaluation of forestry actions. Computer-aided forest simulation software in general has great opportunity for improvement effectiveness and accuracy of forest planning \citep[p. 210]{davis_2001}. They are advantageous against classical tables since they can consider much more relevant aspects for growth and yield such as tree individual concurrence, species mixtures, complex within-stand structures as well as specific thinning and usage rules (\citealp[p. 3]{hansen_2012} \citealp[p. 93]{muys_2010}).

Right from the beginning of DSS in forestry framework, optimization techniques came to use \citep[p. 16]{kangas_2015}. As optimization procedures were initially developed for efficient allocation of finite resources \citep[p. 271]{davis_2001}, they suit to many decision problems in forestry, a field with naturally scarce resources. Recently, combinations of modern growth and yield simulation software and optimization procedures are in focus of forestry research. Co-application of growth simulators and optimization procedures offers opportunities for practical production planning as it enables foresters to respect long-term changing environmental circumstances such as climate change or nitrogen deposition in their operational short-term decisions \citep[p. 1081]{pretzsch_2008}. A recent example from forest sciences can be found in \citet{yousefpour_2009}. They developed a optimization model able to optimize thinning activity in terms of economic return also taking timber production, carbon storage and biodiversity issues into account. They combine the growth and yield simulation software TreeGrOSS with the \textit{dynamic linear programming} optimization procedure. They used an integrated simulation-optimization approach to estimate the monetary drawback of different scenarios changing in their carbon storage and biodiversity characteristics. Combined simulation-optimization methods are not novel in forestry. The \textit{United States Forest Service} e.g. applies combined methods for practical forest planning since the 1970s \citep[p. 33]{hoganson_2015}. Other countries have also experience in practical application of combined methods. Recently combined methods are practically used e.g. in USA and Finland \citep[p. 41]{hoganson_2015}. Since both growth and yield simulation models as well as optimization methods have developed considerably in the last decade, modern combined methods have opportunity to solve more sophisticated decision problems in forestry (\citealp[p. 16-17]{kangas_2015}; \citealp[p. 93]{muys_2010}).

\section{The role of bio-economy in forestry}
\label{sec:intro:biecon}
Bio-economy is defined as a combination of all economic sectors that refine biologic resources with physical, chemical and biotechnological processes \citep[p. 10462]{debesi_2015}. The bio-economy itself is hence no novel sector but an aggregation of former separately regarded sectors which all base on biomass as major resource. Main advantages of bio-economy as an aggregated sector are i.a. research cooperations of former distinct industries in order to benefit from synergy \citep[p. 1]{auer_2016} as well as commonly evaluated supply chains \citep[p. 1-2]{geldermann_2016}. The cooperation is advantageous from view of forestry as it could improve possibilities of planning. If the supply chains of former distinct smaller firms are comprised to a integrated super-regional supply chain including interactions between the firms, which is one aim of bio-economy, logistical planning for forest enterprises could be simplified. Supplying to a well-prepared integrated logistic system will be advantageous in terms of planning dependability and planning costs. Furthermore, from perspective of forestry, a common balance of required resources provides additional advantages in planning. Using modern simulation techniques, forecasting of wood potential is relatively easy. It is hence often practiced in the context of cluster studies \citep[e.g.][]{bmel_2016}, scientific studies (e.g. chapter \ref{chap:hzb}) or practical mensuration \citep[e.g.][]{bockmann_2004}. Sound knowledge of resources demands for longer time periods will have severe positive effects on intermediate-term forest planning as it enables matching the forecasted wood potential with demands.

With an overall turnover of 2 trillion {\euro} in 2014, covering resources from agriculture, forestry and fishery, the European Union's bio-economy sector is leading in worldwide comparison \citep[p. 221, 223]{elchichakli_2016} with especially Germany playing an important role in that scope \citep[p. 200]{hennig_2016}. Although the share of forestry itself as primary producer amounts only to 2 \% (35 billion {\euro}), forestry plays currently an crucial role for resource supply in the bio-economy sector. Wood and paper manufactures had a turnover of 331 {\euro} in 2014. The bio-based sector is a considerable factor in Germany's national economy as well. The share of bio-based economy on Germany's gross value was approximately 8 \% in 2007 including primary production as well as manufacturing and services \citep[p. 29-30]{efken_2012}. In the national bio-economy strategy, the German Government decided to generally strengthen the bio-based industries by 2030 \citep[p. 15-16]{bmel_2014b}. Innovative bio-based products have opportunity to substitute end- and semifinished products which traditionally base on fossil resources. There is thus, from industrial perspective, demand for woody biomass that could further increase. Novel production methods could enhance significance of forest biomass, in particular of small dimensioned wood, for use in bio-refineries. \citet[p. 49]{ekman_2013} e.g. revealed woody biomass to have great potential for chemical semifinished products. Regarding the political and economical circumstances, the importance of bio-economy sector generally and in particular for forestry is thus expected to increase.

\section{Aim of the thesis}
\label{sec:intro:aim}
Biomass from forests has opportunity to provide a largely carbon-neutral supply of material to the bio-based sector. In times of rising demands on all forest function, the question is \textit{''How to match the resource demands of a rising bio-economy industry with the available wood potential without compromising the concept of sustainability?''}. The answer to this question challenging as resources are limited and wood supply, from planning to distribution, are complex causal networks. I present a couple of applied statistical methods, some directly practically applicable, some primary theoretically interesting, trying approximate the solution of the research question from different perspectives.

\section{Structure of the thesis}
\label{sec:intro:struct}
After the general introduction, I present distinct approaches to overcome the research question and the  	introductory stated problems (chapters \ref{chap:hzb} to \ref{chap:opt}). In chapter \ref{chap:discussion}, I close with a general discussion of all 4 studies.

\subsection{Analyzing status and development of raw wood availability in the European beech-dominated central of Germany}
\label{subsec:intro:struct:hzb}
The first step in a planning process is, as already stated, the actual identification of a problem. If resources are not scarce, they will be no distribution problems. For this, we investigated the current and future availability of European beech (\textit{Fagus sylvatica} [L.]) raw wood, one of the most important wood resource for the increasing bio-economy sector, in the beech-dominated center of Germany \citep[p. 1]{auer_2016}. Chapter \ref{chap:hzb} shows the results of a cluster study analyzing the German federal states Lower Saxony, North Rhine-Westphalia, Hesse, Saxony-Anhalt and Thuringia in terms of their beech wood potential. We used the raw data from the national forest inventory \citep{schmitz_2008}, a systematic large-scale sample inventory, to estimate the availability and spatial distribution of beech raw wood. The future potential was forecasted using the DSS WalpPlaner \citep{hansen_2014}.

\subsection{Biomass functions and nutrient contents of European beech, oak, sycamore maple and ash and their meaning for the biomass supply chain}
\label{subsec:intro:struct:bm}
Biomass functions and nutrient contents are helpful for several decision problems in forestry and bio-economy. Potential of forest sites can only be exploited fully when exhausting of soil is excluded. Calculation of site-specific harvesting potential thus requires reliable informations on biomass and nutrients exports. Biomass functions and nutrient hence help calculating the timber potential for the bio-economy. Both are available for beech and oak (\textit{Quercus robur} [L.] and \textit{Quercus petraea} [Matt.]) but not for sycamore maple (\textit{Acer pseudoplatanus} [L.]) and ash (\textit{Fraxinus excelsior} [L.] which are frequently in mixture with beech \citep{ti_2014}. In chapter \ref{chap:bm}, I presents a regression analysis to calculate biomass functions and nutrient contents for beech, oak, sycamore maple and ash.

\subsection{Modelling the economically viable wood in the crown of European beech trees}
\label{subsec:intro:struct:beech_crowns}
\textbf{Chapter \ref{chap:beech_crowns}} is more interesting.

\subsection{Flexible Global Optimization with Simulated-Annealing}
\label{subsec:intro:struct:opt}
heuristics (huys p. 93)
Diese Netzwerk braucht DSS f�r viele Dinge, ...
Da Resourcen begrenzt aber bioeconomy ansteigen soll, sind Optimierungsmodelle umso wichtiger
Optimization methods in particular have opportunity 
Despite all mentioned advantages, inference based decision support tools play only minor role in Germany. While forest growth and yield simulation softwares are  applied in forest mensuration 
Oprimierungsmethoden werden in der Forstplanung der USA und Finland seit ... angewendet, in D bisher nicht. Hoganson \& Meyer
... Behandlungsoptimierung wird m�glich 
Currently no user front-end...
They are hence interesting for most every thinkable decision in forestry framework. Forest growth simulation based optimization models initially appeared in the 1980s. Early models were rather if-then calculation than 

Optimization is useful to program sustainability \citep[p. 15]{kangas_2015}
OM eignen sich gut f�r die Entscheidungsunterst�tzung, weil sie mehrere Kriterien ber�cksichtigen k�nnen. ... Unsere Anwendung optimier die �kon. Funkt. und ber�cksichtig Nachhaltigkeit als ...

Although the advantages of Decision Support Systems (DSS) are widely known and often discussed in forest-science literature, optimization procedures are recently seldom used in Germany. The forest management nowadays widely bases on expert knowledge and decades of experience. The strategic forest treatment is usually ruled by very detailed and long established forest programs. As the demands for wood as well as for further ecosystem benefits is steadily rising, forest planning becomes increasingly more complex. Looking at formal optimization techniques for decision support of forest activities, as they are already performed in foreign countries, therefore seems to be worthwhile. We present an optimization software as DSS for forest thinning activities on operational level which is specifically adopted to German characteristics.

According to the typically categorization of forest optimization models, we present a \textit{type-1} optimization software. Our software basically consists of 3 elements. The forest growth simulation module, the optimization module and the data warehouse. The tree growth is simulated with the TreeGrOSS which is an long established single-tree based growth and treatment simulation software of the Northwest German Research Institute (NW-FVA). The actual optimization bases on sequential TreeGROSS simulations with iteratively changing thinning intensity settings. The response of every iteration is calculated via dimension and tree species specific rating of the simulated harvested wood volume. To ensure sustainability of the simulations, we defined 2 restrictions. The total standing volume of all stands as well as the standing volume of each stand at the end of the simulation are not allowed to be lower than a predefined standing volume limit under standard treatment circumstances. The optimization model additionally enables definition of annual minimum harvesting volumes. As we could not find any suitable optimization algorithm, we developed a novel model basing on the ?Simulated Annealing? method. Our software is specifically parameterizable to the very complex data structure of the TreeGrOSS software. Our optimization software is already freely available as a package for the statistical software R via the Comprehensive R Archive Network. The forest stand data must be stored in a PostgreSQL database. Nagel and Hansen (NW-FVA) developed a software able to translate forest inventory data into TreeGrOSS stands.
We present the core-element
\textbf{Chapter \ref{chap:opt}} is probably most interesting chapter for many readers.
\textsc{\begin{figure}
		\center
		\resizebox{0.7\linewidth}{!}{% Define block styles
\tikzstyle{decision} = [diamond, draw, 
text width=5.5em, text badly centered, node distance=3cm, inner sep=0pt]
\tikzstyle{block} = [rectangle, draw, 
text width=8em, text centered, rounded corners, minimum height=6em]
\tikzstyle{line} = [draw, -latex']
\tikzstyle{cloud} = [draw, ellipse, node distance=10cm,
minimum height=2em]
\tikzstyle{db} = [cylinder, draw, shape border rotate=90, minimum height=6em, text width=8em, text centered,
aspect=0.25]


\begin{tikzpicture}[node distance = 4cm, auto]
    % Place nodes
\node [block] (dev) {interface};
\node [block, left of=dev] (sim) {growth and yield simulation};
\node [block, right of=dev] (opt) {optimizer};
\node [db, below of=sim, node distance=3.3cm] (db) {data-\\warehouse};
    
% Draw edges
\path [line] (sim) -- (db);
\path [line] (db) -- (sim);
\path [line] (sim) -- (dev);
\path [line] (dev) -- (sim);
\path [line] (dev) -- (opt);
\path [line] (opt) -- (dev);

\end{tikzpicture}}
		\caption{Example flowchart. The user front-end is missing.}
		\label{fig:Introduction:flowopt}
	\end{figure}
}