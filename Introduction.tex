\chapter{Introduction}
\label{chap:Introduction}

Since \citet{Carlowitz_1713} introduced the principle of sustainability to forestry, it plays a central role in there, and over the last centuries it has been further developed and extended. To achieve and maintain sustainability in its different specifications \citep{Speidel_1984, Schanz_1996} can be seen as one of the main goals or even the main goal of forestry. A prerequisite for such a sustainable forestry is information on the forest resources, their conditions and changes. This information is usually gained through forest inventories.

%%%%%%%%%%%%%
%% Package %%
%%%%%%%%%%%%%

Bio-economy is defined as
which is not novel in forestry
it means novel industries
meaning today
outlook
Forschungsfrage: Mittel, die gestiegenen Anforderungen an den Wald zu bewaeltigen. Wie koennen die Potenziale des Waldes ausgeschoepft werden, ohne die Nachhaltigkeit zu gefaehrden?
\textsc{\begin{figure}
	\center
	\resizebox{0.7\linewidth}{!}{% Define block styles
\tikzstyle{decision} = [diamond, draw, 
text width=4.5em, text badly centered, node distance=3cm, inner sep=0pt]
\tikzstyle{block} = [rectangle, draw, 
text width=6em, text centered, rounded corners, minimum height=4em]
\tikzstyle{line} = [draw, -latex']
\tikzstyle{cloud} = [draw, ellipse, node distance=3cm,
minimum height=2em]
\tikzstyle{db} = [cylinder, draw, shape border rotate=90, minimum height=4em, text width=6em, text centered,
aspect=0.25]


\begin{tikzpicture}[node distance = 3.5cm, auto]
    % Place nodes
\node [block] (dev) {device};
\node [block, left of=dev] (sim) {growth simulation};
\node [block, right of=dev] (opt) {optimizer};
\node [db, below of=sim, node distance=2.5cm] (db) {database};
    
% Draw edges
\path [line] (sim) -- (db);
\path [line] (db) -- (sim);
\path [line] (sim) -- (dev);
\path [line] (dev) -- (sim);
\path [line] (dev) -- (opt);
\path [line] (opt) -- (dev);

\end{tikzpicture}}
	\caption{Example flowchart.}
	\label{fig:Introduction:flowopt}
\end{figure}
}

Reference example to chapter \ref{chap:hzb}.
