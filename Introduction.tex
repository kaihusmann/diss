\setchapterpreamble[uc][.75\textwidth]{%
	\dictum[Davis, Johnson, Howard and Bettinger, \textit{Forest Management}]{%
		``Forest management, whether for timber production, biodiversity, or any other goals, requires decisions that are based on both our knowledge of the world and human values.''}\vskip1em}

\chapter{Introduction}
\label{chap:Introduction}
Decision making is the last step in a process of planning, starting with actually discovering existence of a decision problem with at least two distinct options \citep[p. 3-4]{kangas_2015}. 

Describe Decisions generally

Following classical theory, 
studies on decision making can be of descriptive or normative type \citep[p. 6]{bitz_2005}. Examining decisions from descriptive perspective means evaluating individual and social actions. Descriptive decision theory analyses realistic decisions with aim of examining how decision-makers act in reality and how decision making actually works. It analyses the principles of decision making descriptively without further investigation. It is not said that lessons learned from empiric-descriptive analysis of decisions lack reasonability. Descriptive decision theory just not inquiries rational behind decisions. Task of the normative decision theory is searching the phenomena that explain reasons behind actual decisions. Normative decision theory only 


As understanding decisions fully comprises knowledge of ... as well as reasonings

without 
knowing what actually drives

OR thst comb. both

 
 states that decisions have reasons without 

fundamental social inquiry

behavior of decision-makers to examine how
 Analyzing decisions normatively 


coexist

distinguished into fundamentally differing types.
there is a rational behind it




how can this help decision-makers in practical issues of forestry ... 

describing decisions in forestry

understanding sense and purpose of decisions helps

Decisions in forest management never affect things solely. Once made, management decisions will have impact on lots of economic, ecologic and social issues.

...Entscheidungen (OptMeth Skript)
Nachhaltigketit
-> Es ist schwer in der FW Ents. zu treffen
Wir zeigen, wie Entscheidungen untertst�tzt werden k�nne. Insbesondere wichtig weil anford an wald steigen
K�nnen menshcl. Entscheid. nicht ersetzen aber unterst�tzen
Since \citet{Carlowitz_1713} introduced the principle of sustainability to forestry, it plays a central role in there, and over the last centuries it has been further developed and extended. To achieve and maintain sustainability in its different specifications \citep{Speidel_1984, Schanz_1996} can be seen as one of the main goals or even the main goal of forestry. A prerequisite for such a sustainable forestry is information on the forest resources, their conditions and changes. This information is usually gained through forest inventories.

\section{The role of bio-economy in foresty}
Bio-economy is defined as
which is not novel in forestry
it means novel industries
meaning today
outlook
Forschungsfrage: Mittel, die gestiegenen Anforderungen an den Wald zu bewaeltigen. Wie koennen die Potenziale des Waldes ausgeschoepft werden, ohne die Nachhaltigkeit zu gefaehrden?
\textsc{\begin{figure}
	\center
	\resizebox{0.7\linewidth}{!}{% Define block styles
\tikzstyle{decision} = [diamond, draw, 
text width=5.5em, text badly centered, node distance=3cm, inner sep=0pt]
\tikzstyle{block} = [rectangle, draw, 
text width=8em, text centered, rounded corners, minimum height=6em]
\tikzstyle{line} = [draw, -latex']
\tikzstyle{cloud} = [draw, ellipse, node distance=10cm,
minimum height=2em]
\tikzstyle{db} = [cylinder, draw, shape border rotate=90, minimum height=6em, text width=8em, text centered,
aspect=0.25]


\begin{tikzpicture}[node distance = 4cm, auto]
    % Place nodes
\node [block] (dev) {interface};
\node [block, left of=dev] (sim) {growth and yield simulation};
\node [block, right of=dev] (opt) {optimizer};
\node [db, below of=sim, node distance=3.3cm] (db) {data-\\warehouse};
    
% Draw edges
\path [line] (sim) -- (db);
\path [line] (db) -- (sim);
\path [line] (sim) -- (dev);
\path [line] (dev) -- (sim);
\path [line] (dev) -- (opt);
\path [line] (opt) -- (dev);

\end{tikzpicture}}
	\caption{Example flowchart.}
	\label{fig:Introduction:flowopt}
\end{figure}
}

Reference example to chapter \ref{chap:hzb}.
