\chapter*{Summary}
\label{chap:Summary}
\addcontentsline{toc}{chapter}{Summary}
Biomass from forests has the potential to provide a largely carbon-neutral supply of material to the bio-based sector and could therefore make a significant contribution to a clean bio-based industry. Modern utilization techniques enable the substitution of fossil resources by renewable biological resources. In this context, the bio-economy contributes to reducing the dependency upon fossil raw material and thus to the reduction of carbon dioxide emissions. The forest sector, as an important primary producer of renewable resources for the bio-economy, plays an important role for the success of the bio-economy. The rising demand for wood, however, could exceed the sustainable achievable wood potential of specific assortments. Effective distribution of the scarce resources is, therefore, a major challenge, which the forest sector has to face. In times of simultaneously rising demands on the forests, the challenge is \textit{''How to distribute the available wood potential effectively without compromising the concept of sustainability answering the question if the resource demands of the rising bio-economy sector can be matched with the available wood potential?''} I present differentiated applied statistical models which strengthen distinct decisions according the woody biomass supply from the perspective of forestry and bio-economy. A descriptive analysis of the availability of woody biomass in central Germany builds the empirical basis of the thesis (Chapter \ref{chap:hzb}). Three explicit examples for the support of distinct decision problems are presented.

Biomass functions and nutrient contents have basically two advantages for the forestry and the bio-based sector (Chapter \ref{chap:bm}). They can help in evaluation and hence exploiting the available wood potential, in particular for small wood and they can support decisions on the raw material supply chain. They enable estimation of single tree dry biomass via easily measurable tree attributes and therefore allow valuation of the biomass flow in the supply chain and assessment of the forest stand individual harvesting potential. Biomass functions for the main tree species can be found in the literature. For other tree species, like sycamore or ash, however, there are only few and very specific studies available. The first decision support method is therefore a set of biomass functions and nutrient contents for European beech, oak, ash and sycamore. It is shown in a test stand that the usage of oak biomass functions for the biomass estimation of sycamore and ash, as it is practiced today, leads to a massive overestimation of the stand biomass. The share of species-rich broadleaf forest stands, and thereby the importance of tree specific biomass functions, is increasing. The introduced models can help to exploit the huge biomass potential of those deciduous mixed stands.

The wood potential of a tree basically consists of the stem as well as the economically viable wood volume in the crown (Chapter \ref{chap:beech_crowns}). Due to the high morphological variability of European beech crowns, taper models are often not satisfactory for predicting the economically viable wood volume arising from crowns. Prediction models with a higher precision are recently still lacking. Aim of the second decision support method is thus the development of a prediction model for the economically viable crown wood volume of European beech trees. It is shown that the economically viable wood volume in the crown significantly depended on the morphological crown type of European beech trees. The presented model calculates the full economically viable wood potential of European beech trees. To make the results applicable in practice forestry, the modeling results were used to develop a regression formula able to forecast the economically viable wood volume in the crown. The model forecasts the total harvestable wood volume very precisely. I can be used to assess the full wood potential and therefore enhance the forecasting of volume flows in the biomass supply chain.

Although combined forest growth and yield si\-mu\-la\-tion-op\-ti\-mi\-za\-tion procedures are already important planning tools in an international framework and have shown their potential to support operational planning, while simultaneously considering long-term issues of, they have so far only played a minor role in Germany (Chapter \ref{chap:opt}). The \textit{Tree Growth Open Source Software} (TreeGrOSS) is a widely used and long established forest growth and yield simulation software of the Northwest German Research Institute which is already used in the intermediate-term forest planning of Northern Germany. In the third example, a harvesting treatment intensity optimization module for the TreeGrOSS packages is presented. The software enables calculation of the optimal thinning intensity in time horizons of up to 20 years, and takes sustainability and the strategic orientation of a forest enterprise into account. It can support the intermediate-term planning of forest enterprises and enhances collaboration between the forestry and the bio-economy sector. It is shown in a case study that the si\-mu\-la\-tion-op\-ti\-mi\-za\-tion model can improve the monetary return of harvesting operations. It can, in addition, be used to calculate the opportunity costs of binding delivery contracts between forest enterprises and wood processing companies. Forest owners can decide whether the benefits in intermediate-term planning justify the opportunity costs of contractual agreed continuous wood delivery amounts. The findings can be used to examine adequate wood prices. The calculations can build an objective, evidence-based fundament for negotiation of intermediate-term delivery contracts between forest enterprises and bio-economy companies.