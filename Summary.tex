\chapter*{Abstract}
\label{chap:Summary}
\addcontentsline{toc}{chapter}{Abstract}
Biomass from forests has the potential to provide a largely carbon-neutral material supply for the bio-based sector and could therefore make a significant contribution to a clean bio-based industry. Modern utilization techniques enable the substitution of fossil resources by renewable biological resources. In this context, the bio-economy contributes to reducing the dependency upon fossil raw material and thus to the reduction of carbon dioxide emissions. The forest sector, as an important primary producer of renewable resources for the bio-economy, plays an important role for the success of the bio-economy. The rising demands for wood, however, could exceed the sustainable achievable wood potentials of specific wood assortments. Effective distribution of the scarce resources is therefore a major challenge, which the forest sector has to face. In times of simultaneously rising demands on the forests, the challenge is \textit{''How to use the available wood potential effectively without compromising the concept of sustainability answering the question if the resource demands of the rising bio-economy sector can be matched with the available wood potential?''} I present differentiated applied statistical models which strengthen distinct decisions according the supply of bio-economy companies with woody biomass from forestry. A descriptive analysis of the availability of woody biomass in central Germany builds the empirical basis of the thesis (Chapter \ref{chap:hzb}). Three explicit examples for the support of distinct decision problems are presented.

Biomass functions and nutrient contents have basically two advantages for the wood supply of the bio-economy sector with biomass from forestry (Chapter \ref{chap:bm}). They can help evaluating and hence exploiting the available wood potential of forest stands and they can support decisions in the raw material supply chain. They enable estimation of single tree dry biomasses via easily measurable tree attributes. Biomass functions and nutrient contents for the main tree species can be found in the literature. For other tree species, like sycamore or ash, however, there are only few and very specific studies available. The first decision support method is therefore a set of biomass functions and nutrient contents for European beech, oak, ash and sycamore. It is shown in a test stand that the usage of oak biomass functions for the biomass estimation of sycamore and ash, as it is practiced today, leads to a massive overestimation of the stand specific biomass. The share of species-rich deciduous forest stands, and thereby the importance of tree specific biomass functions, is increasing. The introduced models can help to exploit the huge biomass potential of those deciduous mixed stands.

The wood potential of a tree basically consists of the stem as well as the economically viable wood volume in the crown (Chapter \ref{chap:beech_crowns}). Due to the high morphological variability of European beech crowns, taper models, which are nowadays often practiced, are not satisfactory for predicting the economically viable wood volume arising from crowns. Prediction models with a higher precision are recently still lacking. The second introduced method is a computer aided package, able to predict the economically viable wood volume arising from crowns of European beech trees. It is shown that the economically viable wood volume in the crown significantly depends on the morphological crown type of European beech trees. The presented model calculates the full economically viable wood potential of European beech trees. It, however, requires very intensive and complicated morphological measurements of specific crown branches. The model is therefore not usable in the framework of the usual forest inventory. To make the results applicable for practitioners, the modeling results are used to develop a regression formula able to forecast the economically viable wood volume in the crown. The model forecasts the total harvestable wood volume very precisely. It can be used to estimate the full wood volume arising from harvesting operations accurately. It therefore enhances the reliability of the entire wood supply chain.

Although combined forest growth and yield si\-mu\-la\-tion-op\-ti\-mi\-za\-tion procedures are already important planning tools in an international framework and have shown their potential to support operational planning, while simultaneously considering long-term issues, they have so far only played a minor role in Germany (Chapter \ref{chap:opt}). The \textit{Tree Growth Open Source Software} (TreeGrOSS) is a widely used and long established forest growth and yield simulation software of the Northwest German Research Institute which is already part of the intermediate-term forest planning of Lower Saxony. The third developed method for the effective usage of the available wood potential is an optimization add-on for the TreeGrOSS packages. A software, able to optimize the harvesting treatment intensity of TreeGroSS simulations is presented. The software enables calculation of the optimal harvesting intensity in time horizons of up to 20 years, and takes sustainability and the strategic orientation of a forest enterprise into account. The software was developed to support the intermediate-term planning of forest enterprises and to enhance collaboration between the forestry and the bio-economy sector. It is shown in a case study that the si\-mu\-la\-tion-op\-ti\-mi\-za\-tion model can, under given circumstances and restrictions, calculate the forest development with the best monetary return. It can, in addition, be used to calculate the opportunity costs of binding delivery contracts between forest enterprises and wood processing companies. Forest owners can hence decide whether the benefits in intermediate-term planning justify the opportunity costs of contractual agreed continuous wood delivery amounts. The model results can build an objective basis for negotiation of intermediate-term delivery contracts between forest enterprises and bio-economy companies.