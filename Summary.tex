\chapter*{Abstract}
\label{chap:Summary}
\addcontentsline{toc}{chapter}{Abstract}
Biomass from forests has the potential to provide a largely carbon-neutral material supply for the bio-based sector and could therefore make a significant contribution to a clean bio-based industry. Modern utilization techniques enable the substitution of fossil resources by renewable biological resources. In this context, the bio-economy contributes to reducing the dependency upon fossil raw material and thus to the reduction of carbon dioxide emissions. The forest sector, as an important primary producer of renewable resources, plays an important role for the success of the bio-economy. The rising demands for wood, however, could exceed the sustainable achievable wood potentials. This would substantially increase the competition on the wood marked. In times of rising demands on the forests, the question is \textit{''How to estimate the sustainably available wood potential from forestry reliably answering the question if the resource demands of the rising bio-economy sector can be matched with the available wood potential?''} I present differentiated applied statistical models able to estimate available wood potentials on different temporal and spatial scales. All models enable decision makers to estimate available resources from forestry very precisely. The estimations are interesting for mainly two reasons. Firstly, reliable and precise calculations of usable wood from forest operations may uncover recently unused potentials. Reliable and accurate estimates of the expectable wood volume from forest usage, furthermore, strengthens calculation of the entire resource supply chain. Those enhanced estimations are, therefore, not only relevant for forest enterprises but for any company being involved in the resource flow.

A descriptive analysis of the availability of woody biomass in central Germany builds the empirical basis of the thesis (Chapter \ref{chap:hzb}). Three explicit examples for the support of distinct decision problems are presented.

Biomass functions and nutrient contents have basically two advantages for the wood supply of the bio-economy sector with biomass from forestry (Chapter \ref{chap:bm}). They can be used to evaluate and hence exploit the available wood potential of forest stands fully and they can strengthen the estimation accuracy of raw material flows. The biomass potential of a forest can only be utilized to an extent that, in the long-term, won't deplete the supply of plant available nutrients in the forest ecosystem. Using easily measured input data, biomass functions allow for a reliable prediction of tree species- and tree fraction-specific single-tree biomasses. In combination with nutrient content data, the site specific ecologically sustainable level of forestry use can be assessed and the site-specific wood utilization potential can be fully exploited. Furthermore, they can easily be applied to estimate biomass amounts in the biomass supply chain. Biomass functions and nutrient contents for the main tree species can be found in the literature. For other tree species, like sycamore or ash, however, there are only few and very specific studies available. The first decision support method is therefore a set of biomass functions and nutrient contents for European beech, oak, ash and sycamore. It is shown in a case study that the usage of oak biomass functions for the biomass estimation of sycamore and ash, as it is practiced today, leads to a massive overestimation of the stand specific biomass. The share of species-rich deciduous forest stands, and thereby the importance of tree specific biomass functions, is increasing. The introduced models can help to exploit the huge biomass potential of those deciduous mixed stands.

The wood potential of a tree basically consists of the stem wood volume as well as the economically viable wood volume in the crown (Chapter \ref{chap:beech_crowns}). Due to the high morphological variability of European beech crowns, taper models, which are nowadays often applied for wood volume estimation, are not satisfactory for predicting the economically viable wood volume arising from crowns. Prediction models with a higher precision are recently still lacking. The second introduced method is a computer aided model, able to predict the economically viable wood volume arising from crowns of European beech trees. It is shown that the economically viable wood volume in the crown significantly depends on the morphological type of European beech crowns. The presented model calculates the full economically viable wood potential of European beech trees with regart to this morphological types. The model, however, requires very intensive and complicated morphological measurements of specific crown branches. The model is therefore not usable in the framework of the practical forest inventory. To make the results applicable for practitioners, the modeling results are used to develop a regression formula able to forecast the economically viable wood volume in the crowns of European beech trees. The model forecasts the total harvestable wood volume very precisely. It can be used to estimate the full wood volume arising from harvesting operations accurately. It therefore enhances the reliability of wood volume estimations for the entire wood supply chain.

Although combined forest growth and yield si\-mu\-la\-tion-op\-ti\-mi\-za\-tion procedures are already important planning tools in an international framework and have shown their potential to support short-term operational planning, while simultaneously considering long-term issues, they have so far only played a minor role in Germany (Chapter \ref{chap:opt}). The \textit{Tree Growth Open Source Software} (TreeGrOSS) is a widely used and long established forest growth and yield simulation software of the Northwest German Research Institute which is already part of the intermediate-term forest planning. The third developed method for the effective usage of the available wood potential is an optimization add-on for the TreeGrOSS packages. A software, able to optimize the harvesting treatment intensity, simulated via TreeGroSS, is presented. The software enables calculation of the optimal harvesting intensity in time horizons of up to 20 years, and takes sustainability and the strategic orientation of a forest enterprise into account. The software was developed to support the intermediate-term planning of forest enterprises and to enhance collaboration between the forestry and the bio-economy sector. It is shown in a case study that the si\-mu\-la\-tion-op\-ti\-mi\-za\-tion model can calculate the forest development with the best monetary return under the given restrictions. It can, in addition, be used to calculate the opportunity costs of binding delivery contracts between forest enterprises and wood processing companies. Forest owners can use the model results to decide whether the benefits in intermediate-term planning justify the opportunity costs of contractual agreed continuous wood delivery amounts. The model results can build an objective basis for negotiation of intermediate-term delivery contracts between forest enterprises and bio-economy companies.