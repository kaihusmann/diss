\chapter{Conclusions}
\label{chap:discussion}
\newpage
\noindent
The thesis aims on evaluation, development and application of proper methods to strengthen the natural raw material supply of upcoming bio-economy industries. It is shown that the success of a greener bio-based industry does, not least, depend on decisions that are made by foresters. Supporting forest management decisions is therefore beneficial not only for the forest sector itself but also for the wood processing industry. If forest enterprises and bio-economy industries were not able to contractually agree on continuous wood supply amounts, the success of promising innovative bio-economy industries could be endangered. Forest decision-makers must decide about the distribution of the available wood potential very carefully as the downstream value creation process of the wood processing industry is much higher than the value creation in the forest sector itself \citep[p. 221,223]{elchichakli_2016}. In a scenario of wood scarcity, distribution decisions can have substantial consequences for the wood processing industry which depends on wood as raw material. This reinforces the importance of DSS in the forest management decision process as they can be used to structure the entire planning process into solvable sub-problems. This, however, also implies the biggest challenge of DSS development. Numerous DSS can already be found the scientific and practical literature. opportunities are, owing to increasing computer power, steadily rising \citep[p. 1065-1067]{pretzsch_2008}. The challenge is hence to find proper and novel methods for relevant decision problems. The benefits, disadvantages and findings of the distinct developed statistical models are discussed in the following.

\section{Findings of the thesis}
\label{sec:discussion:findings}
The introductory stated hypothesis, that matching demands of the rising bio-economy is actually problematic, is verified in chapter \ref{chap:hzb}. Scarcity of woody biomass and the high complexity of the entire wood supply chain are problems forest enterprises and bio-economy companies have to cope with. The descriptive analysis of the recent wood potential and the wood usage reveals significance of the research question and thus also confirms need for profound systems to support the expected raw wood distribution problem. After the relevance of the introduction question is confirmed in chapter \ref{chap:hzb}, the normative analysis in chapters \ref{chap:bm} to \ref{chap:opt} examine particular rationales why decision-makers may not exploit the wood potential fully and how optimal decisions can be calculated.

\subsection{Analyzing status and development of raw wood availability in the European beech-dominated central Germany}
\label{subsec:discussion:struct:hzb}
The available wood potential of all European beech wood assortments was almost completely used in the period between 2002 and 2012 in central Germany (chapter \ref{chap:hzb}). Though the European beech wood potential is expected to rise due to the ongoing forest development programs, the competition situation on the wood market will probably rise. The only perspective for upcoming bio-economy companies to establish on the wood market is hence to get in competition with existing market participants. A detailed analyses of the wood market is therefore a prerequisite for the success of wood processing companies. 

The added value of stem wood is higher than those of any other assortment \citep{nagel_2008}. The stem wood supply is recently entirely exhausted by long-established wood processing companies. It is therefore very difficult for novel companies to establish immediately on the stem wood market. The share of smaller, low-valued European beech wood assortments is about 60 \% (chapter \ref{chap:hzb}). Although this low-valued wood is recently almost entirely used as well, it seems to be partially reachable in future. Approximately half of the wood potential in smaller dimensions is directly energetically used. If a bio-economy company is able to compete with the industrial wood and firewood prices, a solid base of renewable biomass from forests will be at disposal. This would be beneficial for the forest sector as well because the added value of sorting will increase \cite[p. 67]{mohring_1997}.

Two lessons should be learned from the predicted wood scarcity. First of all, the sustainable reachable wood potential should be estimated as exact as possible in order to investigate sources properly. Exacter estimations could uncover recently unused resources. Secondly, it should be ensured that the available potential is distributed thoroughly with view to the actual situation and also to the consequences.

\subsection{Biomass functions and nutrient contents of European beech, oak, sycamore maple and ash and their meaning for the biomass supply chain}
\label{subsec:discussion:struct:bm}
The findings from chapter \ref{chap:bm} can improve gathering the optimal biomass potential of mixed broadleaf forest sites. Comparison of several existing models revealed that estimation accuracy of biomass and nutrient contents in mixed broadleaf tree species sites can be significantly improved by the introduced models. This helps exploiting the site-specific potential fully. The essay is particular relevant for the bio-based industry as innovative productions often require biomass from broadleaf tree species \citep[p. 1]{auer_2016} and the frequency of mixed broadleaf stands steadily increases \citep{ti_2014}. The models enhance the accuracy of tree-specific biomass estimations. They can thus strengthen planning accuracy of entire biomass supply chains.

\subsection{Modelling the economically viable wood in the crown of European beech trees}
\label{subsec:discussion:struct:beech_crowns}
The models from chapter \ref{chap:beech_crowns} aim on assessment of the optimal wood volume in the crowns of European beech. They can hence be used to predict the potential of smaller wood assortments fully. Allometric analysis revealed great wood potentials in the crowns of broadleaf trees. It is therefore essential to consider the predicted wood volume from the crown as a by-product of stem wood production. As the dimension of the wood assortments plays a minor role for many activities (see subsection \ref{subsec:intro:struct:bm} for more details), gathering further wood potential from broadleaf crowns seems to be interesting for bio-economy companies. The models also allow a precise prediction of harvestable wood volume prior harvesting. They can therefore enhance the forecasting of volume flows in the biomass supply chain.

The model is able to calculate the maximal wood potential from an economic perspective only. It is therefore a purely normative-based model that predicts a rational rather than a realistic wood potential. Further non-economic endogenous influences such as fixed assortments lengths or maximum small-end diameters cannot be considered in the model. It allows to calculate the optimal tree-specific wood amount but it does, of course, not predict actual decisions.

\subsection{Flexible Global Optimization with Simulated-Annealing}
\label{subsec:discussion:struct:opt}
Optimization of forest stands treatments offers opportunity to calculate forest operations with highest marginal return (chapter \ref{chap:opt}). The model can be used to calculate the full economic wood potential of an entire forest enterprise in a time period between 10 and 20 years. It is, in contrast to the other introduced models, a model for the support of intermediate-term decisions. Amongst the calculation of the full economic wood potential, sensitivity analysis of different optimization scenarios seems to be interesting for planning purposes. If delivery agreements with upcoming bio-economy companies or any other recipient lead to opportunity costs, source providers must balance very carefully between advantages and drawbacks. The optimization model enables assessment of those opportunity costs. It allows calculation of the drawbacks a delivery contract can have. The forest decision-maker can use the result to balance between the drawback of delivery contracts and their opportunities in planning security. He or she can decide whether the benefits in intermediate-term planning justify the opportunity costs. Furthermore, wood demander can use the information to calculate an adequate wood prizes.

The combined simulation-optimization method that is used to calculate implies high demands on the optimization algorithm (subsection \ref{subsec:intro:struct:opt}). A robust optimization procedure is prerequisite for a reliable simulation-optimization model. It showed up that stochastic methods with dynamic random structures were able to cope with the specific needs of the combined simulation-optimization method. Development of a specific software for optimization of forest thinning activities was straightforward. Intensive  sensitivity analysis revealed the applicability and the efficiency of the package as optimization element in the simulation-optimization software.

\section{Outlook}
\label{sec:discussion:outlook}
The thesis aims more on development and evaluation of inference based models than on practical implementation into DSS. The next step will be the implementation of models into proper software to providing the findings for scientists, students and practitioners. The WaldPlaner DSS provides a favorable front-end for the models as it is already established in forest science and practice.

All introduced methods basically aim on assessment of the full wood potential of forest stands. They enable improvement of the economic forest function. The models from chapters \ref{chap:bm} and \ref{chap:beech_crowns} help examining the maximal site-specific sustainable wood potential. There are, however, numerous good reasons not to fully exploit the potentials. Conservation or recreational issues, which could decrease the actual harvestable wood volume, cannot be considered. Combinations with other DSS that enable consideration of these forest functions could therefore enhance forecasting of actual wood potentials. These aspects provide further arguments for an implementation of the introduced models into the WaldPlaner. WaldPlaner allows parametrization of different treatment options \citep[p. 90-93]{hansen_2014}. Nature conservation oriented treatments will obligatory have lower yields than economic oriented treatments. If any reasons permit full exploitation of the wood potential, WaldPlaner would forecast the limited yield.

The simulation-optimization software optimizes the stand development with view to economic issues. Consideration of the other two forest function is nevertheless possible. As it was performed by \citet{yousefpour_2009}, conservation and recreational issues can be simplified such that they can be implemented as restrictions into the simulation-optimization software. Restrictions, such like harvesting permissions of old broadleaf trees or minimum standing volumes for specific stand ages, could be included. This could enhance the forecasting accuracy in forest stands with nature conservation regulations. The influence of wood quality on the optimal stand treatment is another relevant issue that can be examined with the simulation-optimization software. To achieve this, an interface with user-specific wood prizes must be implemented.

Cooperation between bio-economy companies as well as between companies and forest enterprises is the most important advantage of the bio-economy sector. The cooperations enable calculation of throughout raw material supply chains. The introduced methods, however, cover only a small part of the biomass supply chain. A front-end with an interface to logistic DSS could therefore link the optimized wood amounts with resource distribution simulations or optimizations.

Vorteile erkannt ...Die Modelle sind ein Vorstoss zur Zusammenarbeit.. 