\chapter{Conclusions}
\label{chap:discussion}
\newpage
\noindent
The thesis aims on evaluation, development and application of proper methods to strengthen the natural raw material supply of upcoming bio-economy industries. An important benefit from computer aided is the possibility to split decision problems into solvable sub-problems. This, however, also implies the biggest challenge of DSS development. Numerous DSS can already be found the scientific and practical literature and opportunities are, owing to increasing computer power, steadily rising. The challenge is hence to find a proper and novel methods for relevant decision problems.

\section{Findings of the thesis}
\label{sec:discussion:findings}
The introductory stated hypothesis, that matching demands of the rising bio-economy is actually problematic, is verified in chapter \ref{chap:hzb}. Scarcity of woody biomass and the high complexity of the entire wood supply chain are problems forestry and bio-economy have to cope with. The descriptive analysis of the recent wood potential and the wood usage revealed significance of the research question and thus also confirms need for profound systems to support the expected raw wood distribution problem. After the relevance of the introduction question is confirmed in chapter \ref{chap:hzb}, the following chapters \ref{chap:bm} to \ref{chap:opt} give explicit solutions to overcome specific distinct aspects of the wood distribution problem without compromising the concept of sustainability.

\subsection{Analyzing status and development of raw wood availability in the European beech-dominated central of Germany}
\label{subsec:discussion:dindings:hzb}
The available wood potential of all European beech wood assortments was almost completely used in the period between 2002 and 2012. Though the European beech wood potential is expected to rise due to the ongoing forest development programs, the competition situation on the wood market will probably enhance. The only perspective for upcoming bio-economy companies to establish on the wood market is hence to get in concurrence with existing market participants. The added value of stem wood is higher than those of any other assortment \citep{nagel_2008}. The stem wood supply is recently entirely exhausted by long-established wood processing companies. It is therefore most probably very difficult for novel companies to establish immediately on the stem wood market. Although the potential of low-valued European beech wood is recently entirely used as well, the wood potential seems to be reachable. It showed up, that approximately 50\% of the wood potential in smaller dimensions were directly energetically used. If a bio-economy company was able to compete with the industrial wood price, considerable wood potentials could be tapped. Given, the value added of production allows to payment of wood prices above the typical fire-wood price, a solid base of renewable biomass from forests would be at disposal for the bio-based industry. Further potential could be gathered, if the price for industrial wood could be disbursed.

The forestry sector plays a crucial role in that scope. The success of a greener bio-based industry does, not at least, depend on forestry decisions. If forest enterprises and bio-economy industries are not able to agree on continuous supply chains, to secure stable 

Sie muss Potenzial fr die sinnvolle Verwendung in der bio-economybereitstellen. Der Erfolg der bio-economy hngt massgeblich von der Rohstoffbereitstellung durch die Forstwirtschft ab. Zwei Folgerungen: 1. Potenzial erhhen. 2. Sinvoll verteilen von geringwertigen Sortimenten.
Also -> DSS sind wichtig

\section{Outlook}
\label{sec:discussion:outlook}
The thesis aims more on development of inference based models than on practical implementation into DSS. The next step is to implement the models into proper software, providing the findings for scientists, students and practitioners.

Hybrif models would be better than simulation software, but recently they dont exist (pretzsch Models for Forest Ecosystem Management: A European Perspective
, p. 81)

Currently no user front-end...

Einhalten von Nachhaltigkeit bei Mssnahmen kann simuliert werden, muss nicht hinterher uber Inventur berechnet werden.

Es wird nur die Nutzfunktion optimiert. die andern beiden Waldfunktionen sind nicht teil der Optimierung.
