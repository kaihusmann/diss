\chapter{Conclusions}
\label{chap:discussion}
\newpage
\noindent
The aim of this thesis is the development, evaluation and application of statistical methods for the support of distinct decisions in raw material supply-chains for the bio-economy sector. It is shown that the success of bio-based companies depends, to some degree, on decisions that are made by foresters as primary producers. Supporting decisions in forest management purposes can, therefore, be beneficial, not only for the forest sector itself but also for wood processing companies. Decisions that affect the availability of the wood potential can cause serious consequences for companies that heavily depend on wood as raw material (Sections \ref{sec:intro:biecon} and \ref{sec:hzb:Konsequenzen}). In this thesis, distinct methods to support decisions in forestry enterprises and bio-economy companies are presented. All aim on the assessment of the full wood potentials in different spatial and temporal scales. This is, of course, also relevant for companies that depend on wood as raw materials. Additional wood potentials will probably reduce the competition on the wood marked. Further potentials should become viable for bio-economy companies.

The success of bio-economy companies, furthermore, depends crucially on the distribution of the available wood potential since many production processes require continuous wood supply quantities. The short- and intermediate-term wood distribution planning is a field where cooperation between forestry and bio-economy seems to have advantages for both sides (Section \ref{sec:intro:biecon}).

This reinforces the importance of DSS in the forest management decision process, as they can be used to structure the planning process of wood harvesting and wood distribution into solvable sub-problems thereby enabling consideration of long-term consequences in short-term decisions \citep[p. 1065-1067, 1081]{pretzsch_2008}. The benefits, disadvantages and results of the distinct statistical models for decision support are discussed in the following.

\section{Findings of the thesis}
\label{sec:discussion:findings}
The introductory hypothesis, that matching demands of the rising bio-economy is actually problematic, was verified in Chapter \ref{chap:hzb}. Scarcity of woody biomass and the high complexity of the entire wood supply chain are problems that forest enterprises and bio-economy companies have to face. A descriptive analysis of the recent wood potentials, the wood usages and the structure of the forests reveals the significance of the research question and thus confirms the need for profound systems to estimate the available wood potentials. 

After confirmation of the relevance of the introductory question in Chapter \ref{chap:hzb}, the normative analysis in Chapters \ref{chap:bm} to \ref{chap:opt} examine particular reasons why decision-makers may not be able to exploit the wood potential fully and how optimal potentials could be calculated.

\subsection{Analyzing status and development of raw wood availability in European beech-dominated central Germany}
\label{subsec:discussion:struct:hzb}
The available wood potential of all European beech wood assortments was almost completely exhausted in central Germany in the period between 2002 and 2012 (Chapter \ref{chap:hzb}). Though the European beech wood potential is expected to rise due to ongoing forest development programs, competition on the wood market will probably rise due to the expected decrease of the coniferous wood potential.

The only way for upcoming bio-economy companies to establish on the wood market is to compete with existing market participants. Detailed knowledge of the available raw material potentials and the wood demand is, therefore, a benefit for the success of wood processing companies. Chapter \ref{chap:hzb} is an example for a comprehensive wood market analysis of an interesting supply region for bio-economy companies.

The added value of stem wood is higher than those of any other assortment \citep{nagel_2008}. The stem wood supply is currently entirely exhausted by long-established wood processing companies. It is therefore very difficult for novel companies to establish immediately on the stem wood market. Smaller-dimensioned, low-valued wood assortments make up about 60 \% of the available European Beech wood (Subsection \ref{subsec:hzb:Ergebnisse:Nachhaltig}). Although this low-valued wood is currently almost entirely used as well, the potential may be reachable in the future. Approximately half of the wood potential of smaller dimension wood is directly used energetically. If a bio-economy company is able to compete with industrial wood and firewood prices, a solid base of renewable biomass from forests will then become available. This would be beneficial for the forest sector as well because the added value of sorting would increase \cite[p. 67]{mohring_1997}.

The competition on the European beech wood market is expected to rise in the long-term. From view of the bio-economy, two strategies could help gathering the required resources for novel productions. The first is to get in competition 
1. Hhere Holzpreise zahlen
2. Mehr Holzpotenzial einschlagen.

First of all, the achievable sustainable wood potential should be estimated as accurately as possible in the future in order to evaluate resources properly. Accurate estimations may uncover currently unused resources. This could



Decisions on distribution of scarce resources can hence have serious consequences for forestry enterprises and for wood processing companies.

\subsection{Biomass functions and nutrient contents of European beech, oak, sycamore maple and ash and their meaning for the biomass supply chain}
\label{subsec:discussion:struct:bm}
The biomass and nutrient contents models from Chapter \ref{chap:bm} can help estimating the ecologically viable wood potential of mixed deciduous forest stands. A comparison of several existing models revealed that the introduced models can significantly improve the estimation accuracy of biomass and nutrient contents in mixed deciduous tree species stands. This might help in identifying, and harvesting, formerly unused potential. The essay is particular relevant for the bio-based sector as innovative production methods often require biomass from deciduous tree species \citep[p. 1]{auer_2016} and the frequency of mixed deciduous stands is increasing steadily \citep{ti_2014}. The models enhance the accuracy of tree-specific biomass estimations. They can thus strengthen the planning accuracy of entire biomass supply chains.

\subsection{Modelling the economically viable wood in the crown of European beech trees}
\label{subsec:discussion:struct:beech_crowns}
The models from Chapter \ref{chap:beech_crowns} are aimed at assessing the optimal wood volume in the crowns of European beech trees. They can be used to predict the full economically viable potential of smaller wood assortments. Allometric analysis revealed large wood potential in the crowns of deciduous trees. It is therefore essential to estimate the wood volume from the crown, a by-product of stem wood production, properly. As the dimension of the wood assortments plays a minor role for many activities of bio-economy companies (Subsection \ref{subsec:intro:struct:bm}), gathering further wood potential from deciduous crowns would seem to be of interest. The models also allow a precise prediction of harvestable wood volume prior to harvesting. They can therefore enhance the forecasting of volume flows in the biomass supply chain.

The model is able to calculate the maximal wood potential from an economic perspective only. It is therefore a normative model that predicts a rational, rather than a realistic, wood potential. Further economic or non-economic endogenous influences, such as fixed assortment lengths, maximum small-end diameters or nature conservation issues, cannot be considered in the model. It enables the calculation of the maximal tree-specific wood amount but it does, of course, not predict actual decisions.

\subsection{Flexible Global Optimization with Simulated-Annealing}
\label{subsec:discussion:struct:opt}
The optimization of forest stand treatments makes it possible to determine those forest operations with the highest marginal returns (Chapter \ref{chap:opt}). The developed simulation-optimization model can be used to calculate the full wood potential of an entire forest enterprise in a time period of between 10 and 20 years. It is, in contrast to the other models mentioned, a model for the support of intermediate-term decisions on a lager spatial scale. Bio-economy companies often need a continuous raw material supply to ensure a continuous production. Contractually agreed continuous wood supply quantities can be a prerequisite for the success of upcoming bio-economy companies \citep[p. 221, 223]{elchichakli_2016}. If delivery agreements with upcoming bio-economy companies, or any other recipient, lead to opportunity costs, source providers must weight the advantages and disadvantages very carefully. The simulation-optimization model offers the possibility the calculate those opportunity costs. The decision-maker can use the result to balance between the drawbacks of delivery contracts and their benefits in terms of planning security. They can decide whether the benefits in intermediate-term planning justify the opportunity costs. Furthermore, wood producer and demander can use the results to examine adequate wood prices. The calculations can build an objective basis for negotiation of intermediate-term delivery contracts between forest enterprises and bio-economy companies. The software can also be beneficial for any other analysis where minimum harvesting amounts are relevant. As forest owners usually have continuous fixed costs, they are interested in continuous yields to keep their enterprises solvent \citep[p. 74]{mohring_2010b}. Forest owners can hence use the software to find a trade-off between the advantages and drawbacks of continuous against unbalanced yields. An explicit example where the introduced simulation-optimization approach might be useful for forest owners, is the evaluation of future investments. If investments led to continuous costs, which may result in the need for higher continuous yields to tackle those costs, the model could be used to calculate the opportunity costs of the implied higher yields. The model could be used to calculate the opportunity costs of future investments, if those investments implied minimum yields within a time period between 5 and 20 years. The model could therefore support the trade-off process between the advantages and drawbacks of future investments.

It was shown that optimization of the harvesting schedule can be advantageous from perspective of forestry and from perspective of wood processing companies (Section \ref{sec:intro:biecon}, Subsection \ref{subsec:opt:appendix:results}). As the implemented loss function in the simulation-optimization software returns the rated wood volume (Subsection \ref{subsec:intro:struct:opt}), the actual optimization process optimizes the harvesting operation planning in terms of revenues and processing costs. Decision makers who follow the optimized stand development schedules will optimize their monetary return within the simulation period, though the actual revenue is not explicitly shown. The return of the loss function is, for technical reasons, normalized such that the actual result of the loss function allows no direct interpretation. It is a one to one transformation of the simulated and rated wood volume.

The implemented sustainability restrictions focus only on the sustainability of the stand volumes. Many other definitions are possible \citep[p. 102]{spellmann_2010}. To interpret the stand volumes of a forest development simulation with standard treatment settings as the sustainable stand volume (Subsection \ref{subsec:intro:struct:opt}) is a model simplification. The idea behind this implementation is that sustainable stand volumes are not static. The stand specific sustainable wood potential depends on factors like tree species and tree age (Figure \ref{fig:hzb:fig4_zuw_nutz}). Advantage of this definition is that the stand specific species and age characteristics directly influence the sustainability restrictions and with them the limits of the optimization process. Disadvantage of this implementation is that the users of the simulation-optimization software must predefine simulation settings which are assumed to lead to a sustainable stand development. The parameterization of a standard scenario is hence a crucial step of the optimization procedure. \citet[p. 90]{hansen_2014} give an overview over the necessary treatment settings of the simulation module (the TreeGroSS packages). In the specific examples of this thesis (Subsections \ref{subsec:opt:examples:forest} and \ref{subsec:opt:appendix:results}), the usual treatment settings of central Germany are expected to represent such a sustainable forest development. The settings based on former cluster studies for the forest development in Germany and were specifically adopted to the typical forest developments in central Germany (Section \ref{sec:hzb:Methodik}; Subsection \ref{subsec:opt:appendix:method}).

Owing to the strategic orientation of most German forest enterprises, aim of the forest management in typical European beech stands in Germany, especially in public-owned forests, is to produce high-valued stem wood \citep{nagel_2008}. In the simulation-optimization software, stem wood volume is rated higher than wood volume of lower-valued assortments like industrial wood. The simulation-optimization model hence automatically supports stem wood production since stem wood is favorable in terms of the revenue, just as it is usually the case in practical forestry. The simulation-optimization software, in its recent form, therefore mainly optimizes the schedule and the intensity of thinning activities rather than the schedule and the intensity of target usage operations. This behavior is confirmed in the second case study (Subsection \ref{subsec:opt:appendix:results}). It was nevertheless shown in two examples that optimization of thinning activities substantially increased the harvestable wood amount. The forecasted overall harvestable wood volume remarkably increased whereas the amount of high-values wood from target usage activities was practically unchanged. Forest enterprises can, to conclude, optimize their monetary return in the simulation period with the introduced optimization software. This optimization is mainly reasoned by the rearranging of the thinning harvesting schedule.

For the same reasons, the software is also interesting for wood processing companies who are interested in this low-valued wood. It was examined that low-valued European beech wood could be scarce in the future because the demand for wood as natural resource increases while the available wood potential decreases due to nature conservation issues (Chapter \ref{chap:hzb}). Integration of the simulation-optimization software into the intermediate-term forest planning could increase the available wood potential.

The combined si\-mu\-la\-tion-op\-ti\-mi\-za\-tion method makes high demands on the optimization algorithm (Subsections \ref{subsec:intro:struct:opt} and \ref{subsec:opt:examples:forest}). As a robust optimization module is a prerequisite for a reliable si\-mu\-la\-tion-op\-ti\-mi\-za\-tion software, several optimization approaches were tested and evaluated. Stochastic methods with dynamic random structures performed best. Only such models were able to cope with the specific requirements of the combined si\-mu\-la\-tion-op\-ti\-mi\-za\-tion method. The best optimization model found was a composition of three optimization strategies \citep{corana_1987, kirkpatrick_1983, pronzato_1984}. All approaches were separately available as software packages. A composition of the methods was, however, unpublished. The development of a specific software package for the optimization of forest thinning activities was therefore straightforward. An intensive sensitivity analysis confirmed the applicability and the efficiency of the package as the optimization element in the si\-mu\-la\-tion-op\-ti\-mi\-za\-tion software.

\section{Outlook}
\label{sec:discussion:outlook}
The next step will be the implementation of the developed and evaluated models into proper software in order to make them applicable for scientists, students and practitioners. The WaldPlaner DSS provides a favorable front-end for the models, as it is already established in forest sciences and practice.

One common aim of all introduced methods is the accurate estimation of the full wood potentials of forest stands with view to distinct economic and nature conservation aspects. The models provide decision support such that decision makers can exploit the potentials of their forest stands fully. There are, however, numerous further reasons not to fully exploit the wood potentials of forests with specific regulations. Additional economic, nature conservation or recreational issues than the mentioned aspects in Chapters \ref{chap:bm}, \ref{chap:beech_crowns} and \ref{chap:opt} are not considered at present. Combinations of the models from Chapters \ref{chap:bm} and \ref{chap:beech_crowns} with other DSS, that allow consideration of further regulations, seems to be worthwhile. These aspects provide further arguments for an implementation of the introduced models into the WaldPlaner. WaldPlaner allows parametrization of different treatments scenarios \citep[p. 90-93]{hansen_2014}. Growth and yield simulations with nature conservation oriented treatments settings will forecast lower yields than economic oriented treatments. Combination of the presented methods from Chapters \ref{chap:bm} and \ref{chap:beech_crowns} with the WaldPlaner would hence enable forecasting of the stand specific wood potential with regard to further economic, nature conservation and recreational issues.

The si\-mu\-la\-tion-op\-ti\-mi\-za\-tion software (Chapter \ref{chap:opt}) optimizes the stand development with view to economic issues. Consideration of aspects of the other two forest functions is nevertheless possible. As it was performed e.g. by \citet{yousefpour_2009}, conservation and recreational issues can be simplified such that they can be implemented as restrictions into the si\-mu\-la\-tion-op\-ti\-mi\-za\-tion software. Restrictions, like harvesting permissions of old deciduous trees or limited harvesting amounts for specific tree species, could be included to enhance the forecasting accuracy in forest stands with specific nature conservation regulations. A further advantage, despite the enhanced accuracy of the forecasting, would be the possibility of performing sensitivity analyses between scenarios with specific regulations and standard scenarios. Such applications would enable the evaluation of opportunity costs of nature conservation and recreational issues, in the same way as it was shown for economic issues in Subsection \ref{subsec:opt:examples:forest}.

The influence of wood quality on the optimal stand treatment is another relevant issue that could be examined with the si\-mu\-la\-tion-op\-ti\-mi\-za\-tion software. To achieve this, an interface with user-specific wood prices for different wood assortments must be implemented. Additionally, consideration of interest rates in the simulation-optimization model appears to be promising. Interest rates of alternative investments may have remarkable effects on decisions in the forest planning. Interest rates can be used as proxy variables to consider capital shortage in econometric models \citep[p. 349-351]{mohring_2010}.

Cooperations between wood-processing companies themselves, as well as between companies and forest enterprises, is one of the most important advantages of the bio-economy sector (Section \ref{sec:intro:biecon}). The assessment and interpretation of complete raw material supply chains appears to be especially interesting. The methods introduced in this thesis were developed to support decisions in specific parts of such supply chains, in order to promote the cooperation of the forest and the bio-economy sector. The statistical methods, however, cover only small, distinct parts of the supply chains. To enable meaningful cooperation, they must be somehow made applicable for decision-makers in forestry, as well as in bio-economy or in logistic companies. When implementing the models into DSS, interfaces to other software will be mandatory in order to take full advantage of their possibilities. A front-end with an interface to logistic DSS could link the optimized wood amounts with resource distribution simulations or optimizations.