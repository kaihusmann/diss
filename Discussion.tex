\chapter{Conclusions}
\label{chap:discussion}
\newpage
\noindent
The essays in chapters \ref{chap:hzb} to \ref{chap:opt} give answer to the question of \textit{''How to match the resource demands of a rising bio-economy industry with the available wood potential without compromising the concept of sustainability?''} in different scopes.

Explain/Reveal the need for DSS, 
Develop elemnts for DSS
Develop a DSS

Ressourcen sind knapp (HZB), DSS also notig, um..
programmable vs. non-p. (see intro) In forest often non-p.. High challenge to classify and to find new fields which can be (now with better pc and methods) programmed now.

Die Modelle sind fertig uns soweit mglich validiert. Der nchste Schritt ist, sie in DSS einzubauen

Hybrif models would be better than simulation software, but recently they dont exist (pretzsch Models for Forest Ecosystem Management: A European Perspective
, p. 81)

Currently no user front-end...

Einhalten von Nachhaltigkeit bei Mssnahmen kann simuliert werden, muss nicht hinterher uber Inventur berechnet werden.

Es wird nur die Nutzfunktion optimiert. die andern beiden Waldfunktionen sind nicht teil der Optimierung.
\section{Findings of the thesis}
\label{sec:discussion:findings}
\section{Outlook}
\label{sec:discussion:outlook}