\chapter{Conclusions}
\label{chap:discussion}
\newpage
\noindent
The thesis aims on evaluation, development and application of proper methods to strengthen the natural raw material supply of upcoming bio-economy industries. This means, forestry decisions that accord to the raw material supply of the bio-economy, are to be supported.

The forestry sector plays a crucial role for the raw material supply of the bio-economy. The success of a greener bio-based industry does, not at least, depend on forestry decisions. If forest enterprises and bio-economy industries were not able to contractually agree on continuous wood supply amounts, the success of promising innovative bio-economy industries could be endangered. As introduced in section \ref{sec:intro:aim}, the wood supply of the bio-economy is challenging for forestry as well as for bio-economy companies. If it comes to an scarcity of relevant wood assortments, the forestry sector must face a decision problem. Forest decision-makers must decide about the distribution of the wood available potential very carefully as the downstream value creation process of the wood processing industry is much higher than the value creation in the forestry itself. In a scenario of wood scarcity, forestry decisions can have substantial consequences.

An important benefit from computer aided is the possibility to split a complex decision problems into solvable sub-problems. This, however, also implies the biggest challenge of DSS development. Numerous DSS can already be found the scientific and practical literature and opportunities are, owing to increasing computer power, steadily rising. The challenge is hence to find a proper and novel methods for relevant decision problems. The benefits, disadvantages and meanings of the distinct developed statistical models are discussed in the following.

\section{Findings of the thesis}
\label{sec:discussion:findings}
The introductory stated hypothesis, that matching demands of the rising bio-economy is actually problematic, is verified in chapter \ref{chap:hzb}. Scarcity of woody biomass and the high complexity of the entire wood supply chain are problems forestry and bio-economy have to cope with. The descriptive analysis of the recent wood potential and the wood usage revealed significance of the research question and thus also confirms need for profound systems to support the expected raw wood distribution problem. After the relevance of the introduction question is confirmed in chapter \ref{chap:hzb}, the normative analysis in chapters \ref{chap:bm} to \ref{chap:opt} examine particular rationales why decision-makers may not exploit the wood potential fully and how optimal decisions can be calculated.

The available wood potential of all European beech wood assortments was almost completely used in the period between 2002 and 2012 (chapter \ref{chap:hzb}). Though the European beech wood potential is expected to rise due to the ongoing forest development programs, the competition situation on the wood market will probably enhance. The only perspective for upcoming bio-economy companies to establish on the wood market is hence to get in concurrence with existing market participants. The added value of stem wood is higher than those of any other assortment \citep{nagel_2008}. The stem wood supply is recently entirely exhausted by long-established wood processing companies. It is therefore most probably very difficult for novel companies to establish immediately on the stem wood market. Although the potential of low-valued European beech wood is recently entirely used as well, the wood potential seems to be reachable. It showed up, that approximately 50\% of the wood potential in smaller dimensions were directly energetically used. If a bio-economy company was able to compete with the industrial wood and firewood price, considerable wood potentials could be tapped. If the value added of bio-economy production allows to payment beyond these prizes, a solid base of renewable biomass from forests would be at disposal for the bio-based industry.

Two lessons should be learned from the predicted scarcity. First of all, the sustainable reachable wood potential should be estimated as exact as possible in order to investigate possible further sources. Secondly, it should be ensured that the available potential is distributed thoroughly with view to the actual situation and to the consequences. Both issues can significantly be facilitated using DSS.

The findings from chapter \ref{chap:bm} can improve gathering the optimal biomass potential of mixed broadleaf forest sites. Comparison of several available models revealed that estimation accuracy of biomass and nutrient contents in mixed broadleaf tree species sites can be significantly improved by the introduced models. This is particular relevant for the bio-based industry as innovative productions often require biomass from broadleaf tree species \citep[p. 1]{auer_2016} and the relevance of broadleaf trees will increase \citep{ti_2014}.

The introduced model, however, only captures one crucial aspect in the assessment of the site-specific wood potential. Boden ...


The models, introduced in chapter \ref{chap:beech_crowns} aim on assessment of the optimal wood volume in the crowns of European beech, thereby enhancing the available potential of smaller wood assortments. Allometric analysis revealed that great wood potential actually lies in the crowns of broadleaf trees. It is therefore essential to consider the predicted wood volume from the crown as a by-product of stem wood production in the operational planning. As the dimension of the wood assortments plays minor role for many activities (see subsection \ref{subsec:intro:struct:bm} for more details), gathering further wood potential from broadleaf crowns seems to be interesting for bio-economy companies.
...

\section{Outlook}
\label{sec:discussion:outlook}
The thesis aims more on development of inference based models than on practical implementation into DSS. The next step is to implement the models into proper software, providing the findings for scientists, students and practitioners.

Hybrif models would be better than simulation software, but recently they dont exist (pretzsch Models for Forest Ecosystem Management: A European Perspective
, p. 81)

Currently no user front-end...

Einhalten von Nachhaltigkeit bei Mssnahmen kann simuliert werden, muss nicht hinterher uber Inventur berechnet werden.

Es wird nur die Nutzfunktion optimiert. die andern beiden Waldfunktionen sind nicht teil der Optimierung.
In dieser Arbeit wurde nur ein ganz kleiner Bereich uberpruft.
